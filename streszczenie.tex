\bigskip
\centerline{\begin{Large}{\bf{Streszczenie pracy magisterskiej}}\end{Large} }
\vskip 1.5em
\centerline{\bf \emph{Porównanie wydajności serwisów RESTful w językach Java i Go}}
\vskip 2.5em

\setstretch{1.3}

Niniejsza praca przedstawia porównanie wydajności serwisów RESTful w dwóch różnych językach programowania.

Na początku autor zajmuje się omówieniem istniejących metod tworzenia aplikacji internetowych. Opisuje architekturę REST i sposób jej wykorzystania do budowania systemów informatycznych z zastosowaniem mikroserwisów. Omawia języki Java i Go, które  wykorzystuje w niniejszej pracy. Opisuje ich historię oraz różnice między nimi podczas tworzenia aplikacji internetowych.

W kolejnych rozdziałach autor przedstawia opis aplikacji stworzonych do przeprowadzenia testów. Prezentuje ich architekturę oraz sposób i przypadki użycia. Omawia też narzędzia wykorzystywane podczas przeprowadzania testów.

Główną część pracy stanowią testy wydajnościowe. Autor przedstawia opis poszczególnych testów z uwzględnieniem podziału na grupy oraz sposób ich przeprowadzenia. W dalszej części prezentuje wyniki  testów w formie wykresów, przeprowadza ich wnikliwą analizę i porównuje wydajności testowanych aplikacji.

W ostatnim rozdziale autor dokonuje podsumowania całej pracy. Porównuje    języki Java i Go pod względem wydajności oraz łatwości tworzenia w nich serwisów RESTful. Autor kończy pracę opinią o roli testów wydajnościowych podczas tworzenia systemów informatycznych. 

\vskip 0.8em
\noindent

\newpage
\bigskip
\centerline{\begin{Large}{\bf{The summary of master’s thesis}}\end{Large}} 
\vskip 1.5em
\centerline{\bf \emph{Performance comparison of RESTful services in Java and Go languages}}
\vskip 2.5em

The present study compares the efficiency of RESTful services in two different programming languages.

At the beginning, the Author discusses the existing methods of creating online applications. He describes the REST architecture and its application to building information systems using microservices. He discusses the Java and Go languages used in this study. He describes their history and their differences in the process of creating online applications.

In the subsequent chapters, the Author describes the applications created for performing the tests. He presents their architecture, directions for use and use cases. He also discusses the tools used during the tests.

Performance tests are the main part of the study. The author describes each test, dividing them into groups, and the testing methods used. In further chapters, he presents the tests results as graphs, thoroughly analyses them and compares the efficiencies of the tested applications.

In the final chapter, the Author summarises the entire study. He compares the Java and Go languages in terms of the efficiency and ease of creating RESTful services with them. The Author concludes the study with his opinion on the role of performance tests in the process of creating information systems.

\vskip 0.8em
\noindent