\documentclass[a4paper,12pt,twoside,openany]{report}
%
% Wzorzec pracy dyplomowej
% J. Starzynski (jstar@iem.pw.edu.pl) na podstawie pracy dyplomowej
% mgr. Błażeja Wincenciaka
% Wersja 3.0 - 10 stycznia 2009
%
\usepackage{polski}
\usepackage[utf8]{inputenc}
\usepackage[pdftex]{graphicx}
\usepackage{tabularx}
\usepackage{array}
\usepackage[polish]{babel}
\usepackage{subfigure}
\usepackage{amsfonts}
\usepackage{verbatim}
\usepackage{indentfirst}
\usepackage[pdftex]{hyperref}


% rozmaite polecenia pomocnicze
% gdzie rysunki?
\newcommand{\ImgPath}{img}

% oznaczenie rzeczy do zrobienia/poprawienia
\newcommand{\TODO}{\textbf{TODO}}


% wyroznienie slow kluczowych
\newcommand{\tech}{\texttt}

% na oprawe (1.0cm - 0.7cm)*2 = 0.6cm
% na oprawe (1.1cm - 0.7cm)*2 = 0.8cm
%  oddsidemargin lewy margines na nieparzystych stronach
% evensidemargin lewy margines na parzystych stronach
\def\oprawa{1.05cm}
\addtolength{\oddsidemargin}{\oprawa}
\addtolength{\evensidemargin}{-\oprawa}

% table span multirows
\usepackage{multirow}
\usepackage{enumitem}	% enumitem.pdf
\setlist{listparindent=\parindent, parsep=\parskip} % potrzebuje enumitem

%%%%%%%%%%%%%%% Dodatkowe Pakiety %%%%%%%%%%%%%%%%%
\usepackage{prmag}   % definiuje komendy opieku,nrindeksu, rodzaj pracy, ...

%%%%%%%%%%%%%%% Strona Tytułowa %%%%%%%%%%%%%%%%%
\title{Porównanie wydajności serwisów RESTful w wybranych platformach programowania}

\author{Marcin Jasion}
\nrindeksu{230338}
% wstawienie zdjecia
\zdjecie{\includegraphics[width=4cm]{\ImgPath/zdjecie.jpg}}

\opiekun{prof. nzw. dr hab. inż. Krzysztof Siwek}
% opcjonalnie: \konsultant{prof. Dzielny Konsultant}
\terminwykonania{1 stycznia 1970} % TODO
\datawydaniatematu{1 stycznia 1970} % TODO
\rokakademicki{1970/1970} % TODO

% zakres pracy
\zakres {\begin{enumerate}
 \item Przegląd istniejących rozwiązań
 \item Projekt i implementacja systemu
 \item Opis testów
 \item Analiza wyników
\end{enumerate}
}

% Podziekowanie - opcjonalne
\podziekowania{\noindent
{\Large }
% \bigskip

% Panu prof. nzw. dr hab. inż. Krzysztof Siwek za umożliwienie wykonania pracy oraz wydatną pomoc i opiekę w czasie jej wykonania.

% \bigskip

% {\raggedleft
% Marcin Jasion

% }

{\raggedleft\vfill\itshape{%
  Składam serdeczne podziękowanie\\
  \bigskip
  Panu prof. nzw. dr hab. inż. K. Siwkowi\\
  za umożliwienie wykonania pracy \\
  oraz wydatną pomoc i opiekę\\ 
  w czasie jej wykonania.\\
  \bigskip
  Marcin Jasion
}\par
}
}

\opinie{%
  \newpage
\begin{center}
 {\large\bf  Opinia} \\
\end{center}

  \newpage
  \newpage
\begin{center}
 {\large\bf  Recenzja } 
\end{center}
}


\begin{document}
\maketitle

\chapter{Wstęp}

\chapter{Przegląd literatury}
\section{Serwisy RESTful}
\subsection{Czym jest serwis RESTful}
\subsection{Mikroserwisy}

\section{Java}
\subsection{Historia i ewolucja języka Java}
% TODO: ZA DUZO
\subsection{Java 8}
\subsection{Biblioteka Spring}
\subsubsection{Spring Boot}
\subsubsection{Spring Data MongoDB}
\subsection{Kontenery aplikacji}
\subsubsection{Tomcat8}
\subsubsection{Jetty9}
\subsubsection{Undertow}

\section{NodeJS}
\subsection{Historia i ewolucja platformy NodeJS}
\subsection{Biblioteka ExpressJS}
\subsection{Biblioteka Mongoose}

\section{Go}
\subsection{Historia i ewolucja języka Go}
\subsection{Biblioteka mgo}

\chapter{Narzędzia wykorzystane do wykonania pracy}
\section{Docker}
\section{MongoDB}
\section{ab - Apache HTTP server benchmarking tool}
\section{Amazon Cloud}

\chapter{Projekt Aplikacji}
\section{Opis}
\section{Testy jednostkowe}
\subsection{Wyniki testów}

\chapter{Testy wydajnościowe}
\section{Opis testów}
\subsection{Baza pusta}
\subsection{Baza niepusta}

\section{Wyniki testów}
\subsection{Baza pusta}
\subsection{Baza niepusta}
\section{Analiza wyników}

\chapter{Wnioski}

\appendix
\chapter{Implementacja serwisu języku Java}
TODO
\chapter{Implementacja serwisu na platformę NodeJS}
\chapter{Implementacja serwisu języku Go}
\chapter{Testy integracyjne}

\begin{thebibliography}{99}
\addcontentsline{toc}{chapter}{Bibliografia}

\end{thebibliography}

\zakonczenie  % wklejenie recenzji i opinii

\end{document}
%+++ END +++
