\documentclass[a4paper,12pt,oneside,openany]{report}
%
% Wzorzec pracy dyplomowej
% J. Starzynski (jstar@iem.pw.edu.pl) na podstawie pracy dyplomowej
% mgr. Błażeja Wincenciaka
% Wersja 3.0 - 10 stycznia 2009
%
\usepackage{polski}
\usepackage[utf8]{inputenc}
\usepackage[pdftex]{graphicx}
\usepackage{tabularx}
\usepackage{array}
\usepackage[polish]{babel}
% \usepackage{subfigure}
\usepackage{booktabs}


% \linespread{1.1}

\usepackage{amsfonts}
\usepackage{verbatim}
\usepackage{indentfirst}
\usepackage[pdftex]{hyperref}
% Java syntax
\usepackage{listings}
\usepackage{color}

\definecolor{dkgreen}{rgb}{0,0.6,0}
\definecolor{gray}{rgb}{0.5,0.5,0.5}
\definecolor{mauve}{rgb}{0.58,0,0.82}


\lstdefinelanguage{JavaScript}{
  keywords={typeof, new, true, false, catch, function, return, null, catch, switch, var, if, in, while, do, else, case, break},
  keywordstyle=\color{blue}\bfseries,
  ndkeywords={class, export, boolean, throw, implements, import, this},
  ndkeywordstyle=\color{darkgray}\bfseries,
  identifierstyle=\color{black},
  sensitive=false,
  comment=[l]{//},
  morecomment=[s]{/*}{*/},
  commentstyle=\color{purple}\ttfamily,
  stringstyle=\color{red}\ttfamily,
  morestring=[b]',
  morestring=[b]"
}
\lstdefinelanguage{Go}{
morekeywords=[1]{break,default,func,interface,%
    case,defer,go,map,struct,chan,else,goto,package,%
    switch,const,fallthrough,if,range,type,continue,%
    for,import,return,var,select},
  % Special identifiers, builtin functions
  morekeywords=[2]{make,new,nil,len,cap,copy,complex,%
    real,imag,panic,recover,print,println,iota,close,%
    closed,_,true,false,append,delete},
  % Basic types
  morekeywords=[3]{%
    string,int,uint,uintptr,double,float,byte,%
    int8,int16,int32,int64,int128,%
    uint8,uint16,uint32,uint64,uint128,%
    float32,float64,complex64,complex128,%
    rune},
  % Strings : "toto", 'toto', `toto`
  morestring=[b]{"},
  morestring=[b]{'},
  morestring=[b]{`},
  % Comments : /* comment */ and // comment
  comment=[l]{//},
  morecomment=[s]{/*}{*/},
  % Options
  sensitive=true
}
\lstset{frame=tb,
  language=Java,
  aboveskip=1cm,
  belowskip=1cm,
  showstringspaces=false,
  columns=flexible,
  basicstyle={\small\ttfamily},
  numbers=left,
  numberstyle=\tiny\color{gray},
  keywordstyle=\color{blue},
  commentstyle=\color{dkgreen},
  stringstyle=\color{mauve},
  breaklines=true,
  breakatwhitespace=true,
  tabsize=4
}


%  Java syntax

% rozmaite polecenia pomocnicze
% gdzie rysunki?
\newcommand{\ImgPath}{img}

% oznaczenie rzeczy do zrobienia/poprawienia
\newcommand{\TODO}{\textbf{TODO}}


% wyroznienie slow kluczowych
\newcommand{\tech}{\texttt}

% na oprawe (1.0cm - 0.7cm)*2 = 0.6cm
% na oprawe (1.1cm - 0.7cm)*2 = 0.8cm
%  oddsidemargin lewy margines na nieparzystych stronach
% evensidemargin lewy margines na parzystych stronach
\def\oprawa{1.05cm}
\addtolength{\oddsidemargin}{\oprawa}
\addtolength{\evensidemargin}{-\oprawa}

% table span multirows
\usepackage{multirow}
\usepackage{enumitem}	% enumitem.pdf
\usepackage[section]{placeins}

\setlist{listparindent=\parindent, parsep=\parskip} % potrzebuje enumitem

%%%%%%%%%%%%%%% Dodatkowe Pakiety %%%%%%%%%%%%%%%%%
\usepackage{prmag}   % definiuje komendy opieku,nrindeksu, rodzaj pracy, ...
\usepackage{pgfplots}
\pgfplotsset{compat=newest}
\usepackage{tikz}
\usepackage{subcaption}
\usepackage{caption}
\usepackage{setspace}
\DeclareCaptionFont{black}{\color{black}}
\DeclareCaptionFormat{listing}{\colorbox{white}{\parbox{\textwidth}{#1#2#3}}}
\captionsetup[lstlisting]{format=listing,labelfont=black,textfont=black}

\renewcommand\lstlistingname{Przykład}

%%%%%%%%%%%%%%% Strona Tytułowa %%%%%%%%%%%%%%%%%
\title{Porównanie wydajności serwisów RESTful w wybranych platformach programowania}

\author{Marcin Jasion}
\nrindeksu{230338}
% wstawienie zdjecia
\zdjecie{\includegraphics[width=4cm]{\ImgPath/zdjecie_profilowe.jpg}}

\opiekun{prof. nzw. dr hab. inż. Krzysztof Siwek}
\terminwykonania{1 stycznia 1970} % TODO
\datawydaniatematu{1 stycznia 1970} % TODO
\rokakademicki{1970/1970} % TODO

% zakres pracy
\zakres {\begin{enumerate}
 \item Przegląd istniejących rozwiązań
 \item Projekt i implementacja systemu
 \item Opis testów
 \item Analiza wyników
\end{enumerate}
}

% Podziekowanie - opcjoalne
% \podziekowania{\noindent
{\Large }
% \bigskip

% Panu prof. nzw. dr hab. inż. Krzysztof Siwek za umożliwienie wykonania pracy oraz wydatną pomoc i opiekę w czasie jej wykonania.

% \bigskip

% {\raggedleft
% Marcin Jasion

% }

{\raggedleft\vfill\itshape{%
  Składam serdeczne podziękowanie\\
  \bigskip
  Panu prof. nzw. dr hab. inż. K. Siwkowi\\
  za umożliwienie wykonania pracy \\
  oraz wydatną pomoc i opiekę\\ 
  w czasie jej wykonania.\\
  \bigskip
  Marcin Jasion
}\par
}
}

\opinie{%
  \newpage
\begin{center}
 {\large\bf  Opinia} \\
\end{center}

  \newpage
  \newpage
\begin{center}
 {\large\bf  Recenzja } 
\end{center}
}
\begin{document}

\maketitle
\setstretch{1.25}
\chapter{Wstęp}

W dobie coraz dynamicznej rozwijającej się sieci Internet serwisy internetowe stają się częścią  naszego życia. Serwisy informacyjne, sklepy internetowe czy bankowość elektroniczna pomagają w codziennych czynnościach,  zaoszczędzając w ten sposób nasz cenny czas i wysiłek. Jednak, aby  serwisy internetowe spełniały swoje funkcje muszą  mieć odpowiednią wydajność. 

Dla użytkownika korzystającego z portali społecznościowych, system z którego korzysta wydaje się być prosty. W rzeczywistości jednak taki system jest bardzo skomplikowanym oprogramowaniem, które często działa na wielu serwerach.

W celu zapewnienia sprawnego funkcjonowania  systemu, jego niezawodności oraz umożliwienia jego ciągłego rozwoju zostało wypracowanych wiele metod inżynierii oprogramowania. Dziedzina ta wytworzyła zasady postępowania podczas tworzenia i rozwijania oprogramowania, które obejmują sposoby gromadzenia wymagań o systemie, jego implementacji, testowania jak również wdrażania. Dobrze zaprojektowany i wykonany system może przynieść ogromne zyski czasowe i finansowe. 

Wydajność jest jednym z aspektów powodzenia funkcjonowania systemu. Zaplanowanie i wykonanie odpowiednich testów wydajnościowych pozwala zbadać jak system zachowa się podczas dużego obciążenia, jak dużo użytkowników może równocześnie z niego korzystać oraz czy spełnia założone wymagania. Dzięki testom można też znaleźć i wyeliminować najsłabsze punkty systemu.

Celem niniejszej pracy jest  porównanie wydajności serwisów  \textsl{RESTful} w wybranych platformach oprogramowania. Na potrzeby badań  stworzone zostaną aplikacje typu \textsl{RESTful} zaimplementowane  w dwóch różnych językach programowania: \textsl{Java} i \textsl{Go}.  Do analizy wydajności serwisów zostaną opracowane i wykonane testy wydajnościowe, na podstawie których zostanie przeprowadzona analiza.

\chapter{Przegląd literatury}

\section{REST}
Termin \textsl{REST}(ang. \textsl{REpresntation State Transfer}) po raz pierwszy został zaprezentowany w pracy doktorskiej "Architectural Styles and te Design of Network-based Software Architectures", której autorem był Roy Thomas Fielding w 2000 roku. Autor pracy zaprezentował \textsl{REST} jako architekturę do tworzenia rozproszonych aplikacji wykorzystujących \textsl{hypermedia} czyli danych będących powiązaniem grafiki, dźwięku, wideo, tekstu oraz \textsl{hyperlinków}). \textsl{REST} łączy kilka wzorców tworzenia aplikacji sieciowych. 

Pierwszym z nich jest klient-serwer. W tym przypadku polega on na podziale ról: serwera i klienta. Rolą serwera jest zapewnienie danych, natomiast rolą klienta jest zgłaszanie się po dane. Dla klientów do zalet takiego rozwiązania należy możliwość tworzenia aplikacji wykorzystujących ten sam interfejs, uruchomionych na różnych systemach. Natomiast dla serwerów pozwala to na rozwijanie interfejsów, niezależnie od klientów. Takie podejście jest szczególnie przydatne w przypadku aplikacji internetowych. Wzorzec klient-serwer jest to najczęściej spotykany wzorzec tworzenia aplikacji sieciowych.

Kolejnym wzorcem jest bez stanowość(ang. \textsl{stateless}). We wzorcu tym ważne jest, by każde żądanie z klienta do serwera musi zawierać informacje potrzebne do zrozumienia tego żądania. W praktyce oznacza to, że klient nie może poprosić serwera o przechowanie w swoim kontekście danych szczególnych temu klientowi. Stan sesji jest wiec trzymany w kliencie. Takie podejście pozwala na poprawienie skalowalności. Jednak powoduje ono, że czasem część danych jest przesyłana nadmiarowo.

W celu poprawy wydajności wykorzystuje się pamięć podręczną(ang. \textsl{cache}). Celem takiego rozwiązania jest, by bez potrzeby nie generować tej samej odpowiedzi dwukrotnie. Pamięć podręczna może zostać zaimplementowana u klienta oraz serwera.

Systemy budowane w oparciu o \textsl{REST} mogą wykorzystywać architekturę warstwową. Dzięki temu podejściu możliwe jest zhierarchizowanie komponentów systemu, tak by nie nie musiały wiedzieć o systemie więcej, niż to do czego zostały stworzone. 

Charakterystyczną cechą odróżniającą \textsl{REST} od innych wzorców aplikacji sieciowych jest nacisk na ujednolicenie interfejsu pomiędzy komponentami. Zastosowanie tej zasady z inżynierii oprogramowania sprawia, że architektura systemu jest uproszona. Przekłada się to na ogólną widoczność interakcji w systemie. Implementacje od usług są rozdzielone, co przekłada się na możliwość niezależnego rozwoju wykorzystywanych serwisów.

\subsection{REST a RESTful}
% https://stackoverflow.com/questions/671118/what-exactly-is-restful-programming
\subsection{HATEOAS}
\section{Mikroserwisy}
Mikroserwisy są małymi, niezależnymi serwisami, które ze sobą współpracują\cite{newman}. Mikroserwisy są odpowiedzią na rosnącą popularność trendów w tworzeniu tworzenia oprogramowania, które są ukierunkowane na domenę(ang. \textsl{Domain-Driven Design}, skrót \textsl{DDD}), zapewniające ciągłość dostaw zmian(ang. \textsl{continous delivery}), tworzone przez małe, niezależne zespoły oraz pozwalające na ich łatwe skalowanie. 

Oprogramowanie ukierunkowane na domenę pierwszy zyskało popularność w 2003. Eric Evans w książce \textsl{Domain-Driven Design: Tackling Complexity in the Heart of Software}\cite{ddd} przedstawił sposób na tworzenie oprogramowania skupionym na domenie. Należy podkreślić że DDD nie jest metodyką tworzenia oprogramowania. Jest sposobem na zrozumienie działania rozległych systemów, co pozwala na ustalenie priorytetów podczas ich tworzenia. Książka przedstawiała to, jak ważna jest reprezentacja prawdziwego świata w tworzonym oprogramowaniu. Od 2003 roku podejście to podlega nieustannemu rozwojowi koncepcyjnemu, a jeszcze technologicznemu.

Podejście do tworzenia oprogramowania przy użyciu mikroserwisów idzie w parze z \textsl{DDD}. Mikroserwisy jak wcześniej wspomniałem są małe. Tworzone oprogramowanie rośnie wraz ze zmianami dodawanymi do systemu. Gdy system rozrośnie się, dodanie kolejnej zmiany może skutkować błędami w innej części systemu, których nie dało się przewidzieć. Często taka sytuacja zdarza się przy dużych systemach, będącymi monolitami.

Mikroserwisy przez to, że są małe pozwalają być skupione tylko na zadaniu, do którego zostały stworzone. Komunikacja między mikroserwisami odbywa się przez, sieć więc bez problemu można je uruchomić na odizolowanym środowisku lub platformie przez co mikroserwisy stają się niezależne. 

 Uruchamianie mikroserwisów na różnych systemach pozwala na stworzenie systemu bardziej odpornego na awarie. Jeśli jeden z mikroserwisów przestanie działać np. przez awarię sprzętu, spowoduje to wyłączenie pojedynczej funkcjonalności. Sam system będzie wciąż działał, a niedziałającą usługę szybko można przenieść i uruchomić w innym miejscu. Przy monolitycznym systemie, by zabezpieczyć się przed awariami musimy uruchomić ten sam system na kilku maszynach marnując przez to zasoby.

Mikroserwisy pozwalają na łatwe skalowanie. Wystarczy że uruchomimy ten sam serwis w kilku miejscach, a wydajność danej części systemu powinna wzrosnąć. W monolitycznych systemach jesteśmy zmuszeni do skalowania całych systemów więc również funkcjonalności, których w systemie rzadziej się używa również będą powielone.

Niezależność mikroserwisów jest również zaletą przy wyborze technologi. Można stworzyć te same mikroserwisy w różnych językach o ile każdy będzie działać tak samo. Przydaje się to również gdy chcemy poprawić wydajność danej usługi lub technologia, której używamy nie jest przystosowana pod konkretne zadanie. 

Kolejną sporą zaletą tworzenia systemów opartych o mikroserwisy jest wygoda ich wdrażania na środowiska produkcyjne niezależnie od pozostałych części systemu. Jeśli dodamy nową zmianę w danych serwisie, nie ma potrzeby restartowania całego systemu. Również w przypadku błędu pozwala to na szybkie znalezienie błędu, poprawienie i wdrożenie bez potrzeby zbędnego oczekiwania. 

\section{Java}
\subsection{Historia i ewolucja języka Java}
Początki języka Java miały miejsce w 1991 roku. Wtedy to Patrick Naughton i James Gosling, inżynierowie firmy Sun,  wpadli na pomysł zaprojektowania języka, który będzie można uruchomić na urządzeniach takich jak dekodery telewizyjne\cite{java8}. Ponieważ urządzenia takie pochodziły od różnych producentów i każdy mógł stosować różne procesory, język nie mógł być oparty na jednej architekturze. W ten sposób zespół pracujący na rozwiązaniem wskrzesił model uruchamiania oprogramowania na maszynach wirtualnych, gdzie programy były kompilowane do kodu pośredniego.  
Początkowo język nosił nazwę Oak. Ponieważ taki język już istniał, postanowiono zmienić nazwę na Java.

Do roku 1994 roku zespół projektowy bez skutku próbował wykorzystać język w urządzeniach kablowych. Wtedy postanowili wykorzystać język do stworzenia przeglądarki internetowej, która będzie niezależna od architektury. W 1995 zaprezentowali oni efekt swojej pracy: przeglądarkę HotJava. Przez wbudowaną przenośność język został uznany, że jest wart rozwijania by w 1996 roku wydano wersję 1.0. Krótko po tym wyszła wersja 1.1 rozszerzająca możliwości m.in o możliwośc drukowania, dodano model zdarzeń dla programowania GUI oraz poprawiono refleksję.

Język miał wciąż spore braki i w 1998 roku wydano wersję 1.2. Wersja ta była ogromną zmianą w porównaniu do poprzedniczek, dlatego zmieniła swoją marketingową nazwę na "Java 2 Standard Edition Software Development Kit Version 1.2". Wraz z nią pojawiło się wiele nowych elementów. Do najważniejszych z nich należą biblioteka Collection oraz Swing. Biblioteka Swing pozwalała na tworzenie aplikacji z graficznymi interfejsami użytkownika(w tym apletów w przeglądarce), której elementy w łatwy sposób skalowały się do ekranów urządzeń na których była uruchamiana. W tym samym momencie swój początek miały dwie wersje języka. Micro Edition przeznaczona do urządzeń przenośnych np. telefonów komórkowych oraz Enterprise Edition do pisania aplikacji serwerowych. Kolejne wersje języka, 1.3 i 1.4 wydane odpowiednio w 2000 i 2002 roku, wprowadzały ulepszenia w samym języku oraz rozszerzały standardową bibliotekę. Z biegiem czasu popularność aplikacji uruchamianych po stronie klienta gasła, jednak rosła popularność pisania aplikacji po stronie uruchamianych na serwerach.

W 2004 pojawiła się nowa wersja języka oznaczona numerem 5.0. Wprowadziła ona typy sparametryzowane(ang. \textsl{generic types}) co rozszerzyło możliwości pisania oprogramowania, bardziej odpornego na błędy programisty. Doszły różwnież rzeczy zaczerpnięte z języka C\#: pętla \textsl{for each} do poruszania się po tabliach i kolekcjach, oraz \textsl{autoboxing} pozwalający konwertować typy proste na obiekty i odwrotnie. Dodane zmiany były zaimplementowane tak, by nie zmieniać nic w maszynie wirtualnej.

Wersja Java SE 6 nie wnosiła nowych funkcji. Rozszerzała ona jednak bibliotekę standardową, oraz poprawiono wydajność i bezpieczeństwo.

W wyniku problemów finansowych firmy Sun, koncern Oracle w 2010 wykupił firmę i w ten sposób stał się właścicielem praw do języka Java. Mocno opóźniona kolejna wersja wyszła w 2011. Wprowadzała ona sporo zmian. Do najważniejszych z nich należało dodanie frameworka \textsl{Fork/Join} upraszający programowanie równoległe, pozwalając na tworzenie procesów oraz poprawa wydajności na systemach z procesorami wielordzeniowymi. Kolejną sporą nowością była nowa biblioteka obsługi wejścia/wyjścia, nazwana NIO. Java 7 wnosiła też mechanizm \textsl{invokedynamic}. Mechanizm ten pozwalał na nowy sposób wywołania metod. Zmiana wymusiła zmiany w \textsl{bytecode} języka. Zmiana miała się stać fundamentem, do tego co przyniesie Java 8.

\subsection{Java 8}
Obecnie istniejącą wersją języka Java jest wersja numer 8. Została opublikowana w 2014 roku przynosząc długo oczekiwane zmiany.\\*
Najważniejszą z nich i najbardziej znaną były wyrażenia \textsl{Lambda}. Wyrażenia te pozwalają na czyste i zwięzłe wyrażenie pojedynczej metody. Wyrażenia te są wykorzystywane często przy operowaniu na kolekcjach między innymi do sortowania czy filtrowania kolekcji.\\*
Na przykładzie \ref{lst:javaanonymous} zaprezentowany jest przykład sortowania listy osób, alfabetycznie po imieniu, przy użyciu klas anonimowych. Przykład \ref{lst:javalambda} prezentuje tą samą funkcjonalność, jednak z wykorzystaniem wyrażeń \textsl{Lambda}. 

\begin{lstlisting}[caption=Sortowanie kolekcji w języku Java przy użyciu klas anonimowych, label={lst:javaanonymous}]
Collections.sort(persons, new Comparator<Person>(){
  public int compare(Person p1, Person p2){
    return p1.firstName.compareTo(p2.firstName);
  }
});
\end{lstlisting}

\begin{lstlisting}[caption=Sortowanie kolekcji w języku Java przy użyciu wyrażeń \textsl{Lambda},label={lst:javalambda}, aboveskip=0mm]
Collections.sort(persons, (p1, p2) -> p1.firstName.compareTo(p2.firstName));
\end{lstlisting}
Kod przy użyciu wyrażeń \textsl{Lambda} jest prostszy i krótszy.\\*
Wyrażenia te w \textsl{bytecode} używają mechanizmu \textsl{invokedynamic} wprowadzonego w Javie 7. Wykorzystanie tego mechanizmu pozwala na opóźnienie przetłumaczenia wyrażenia w \textsl{bytecode} do momentu ich wywołania. Wynikiem wyrażenia \textsl{Lambda} jest metoda statyczna, która jest tworzona w czasie wykonania \textsl{bytecode}.\\*
Kolejną sporą zmianą jest możliwość tworzenia metod domyślnych(ang. \textsl{default methods}) i statycznych(ang. \textsl{static methods}). Pozwoliło to m.in. na dostarczenie implementacji metody, bez potrzeby tworzenia klas abstrakcyjnych.

Dużą nowością w tej wersji były nowe klasy do obsługi dat i godzin. Pozwalają one na operacje takich jak dodawanie czy odejmowanie w sposób uporządkowany i bardziej naturalny do zrozumienia.\\*
Z mniej popularnych nowości jest możliwość wywoływania kodu \textsl{JavaScript} na wirtualnej maszynie Java(ang. \textsl{Java Virtual Machine}, w skrócie JVM).


\subsection{Spring}
\textsl{Spring} jest szkieletem aplikacji(ang. \textsl{framework}), zapoczątkowanym przez Roda Johnsona, który to w 2001 opublikował książkę \textsl{Expert One-on-One: J2EE Design and Development}\cite{jeedesign}. Zaprezentował w niej swoją bibliotekę, która pozwalała tworzyć oprogramowanie biznesowe, składające się się prostych komponentów \textsl{JavaBean}. Biblioteka upubliczniona została natomiast w 2002 roku\cite{springinaction}.

Z czasem biblioteka rozrastała się o nowe funkcję stając się jedną z najpopularniejszych bibliotek do tworzenia aplikacji. Złożenie biblioteki z modułów pozwoliło na dobieranie ich do potrzeb architektury swojej aplikacji. Do najważniejszych i najbardziej popularnych modułów należą:
\begin{itemize}
\item IoC - do wstrzykiwania zależności
\item MVC - do tworzenia aplikacji webowych
\item Data oraz JDBC - jako warstwa dostępu do baz danych
\item Security - autoryzacja i zabezpieczanie aplikacji
\end{itemize}
Najnowsza wersja biblioteki \textsl{Spring} wydana została w czerwcu 2015 i została oznaczona numerem 4.2.

Tworzenie aplikacji opartych o tą bibliotekę było skomplikowane. Wiązało się z tworzeniem wielu plików konfiguracyjnych oraz zapoznaniem dogłębnie z dokumentacją. W 2013 roku, wraz z wyjściem wersji 4 biblioteki powstał projekt \textsl{Spring Boot}. 

\textsl{Spring Boot} jest biblioteką, która pozwala na tworzenie oprogramowania z zachowaniem zasady\textsl{convention over configuration}. Pozwala on na uruchomienie aplikacji, bez potrzeby zbędnego konfigurowania. Zasada ta zakłada domyślną konfiguracje, pozwaląjącą na uruchomienie jej bez potrzeby zbędnego modyfikowania. Dopiero szczegółowe konfigurowanie wymaga interwencji programisty. W \textsl{Spring Boot} zasada ta stała się na tyle rozpowszechniona, że od rozpoczęcia tworzenia aplikacji, do jej pierwszego uruchomienia wystarczy kilka minut. Dodanie kolejnych bibliotek również nie wymaga żmudnych konfiguracji. Tworzenie aplikacji, które będzie można uruchomić w ciągu kilku minut od rozpoczęcia programowania było celem, któremu twórcy \textsl{Spring} musieli stawić czoła w kontekście mikroserwisów. 

\textsl{Spring Boot} obecnie pozwala na użycie wszystkich modułów, jakich \textsl{Spring} dostarcza(m.in. \textsl{MVC}, \textsl{Security}, \textsl{Data}) oraz dodając kolejne(m.in. Jetty, Tomcat, metryki, shell). 

\subsection{Kontenery aplikacji}
\subsubsection{Tomcat8}
\subsubsection{Jetty9}

\section{Go}
\subsection{Historia i ewolucja języka Go}
\subsection{Biblioteka mgo}

\chapter{Narzędzia wykorzystane do przeprowadzenia testów}

\section{Docker}
Docker jest platforma dla programistów i administratorów systemóœ do tworzenia, dostarczania i uruchamiania aplikacji. Docker pozwala zbudować aplikację z zależnościami, która zachowywać się będzie tak samo na środowisku produkcyjnym jak i programistycznym. Dzieje się tak, ponieważ budując aplikację tworzymy obrazy, które po zbudowaniu przenosimy na docelowe środowisko.\\*
Docker opiera się na kontenerach linuxowych(LXC - Linux Containers). Kontenery w Linuxie są wirtualizacją na poziome systemu operacyjnego, która umożliwia na separacje aplikacji od systemu operacyjnego i fizycznej infrastruktury wykorzystywanej m.in. do połączeń sieciowych czy plików. Każdy z kontenerów może uruchomić swój proces, może mieć własnych użytkowników. Kontenerów w jednym systemie może być uruchomiona nieograniczona ilość. Konteneryzacja w przeciwieństwie do wirtualizacji  oferuje niewielki narzut na zasoby. Uruchomienie pojedynczego kontenera ogranicza się do wykonania kilku standardowych poleceń systemowych. 
Dla niniejszej pracy użyto Docker w wersji 1.9.0.

\section{MongoDB}
MongoDB(https://www.mongodb.org/)jest nierelacyjną bazą danych(ang. \textsl{NoSQL database}). Główną cechą tej bazy jest brak ściśle zdefiniowanej struktury. Dane w bazie przechowywane są w postaci dokumentów. Dokument jest strukturą złożoną z par klucz-wartość, przypominających obiekt JSON. Wartości w dokumencie mogą zawierać inne dokumenty, tablice czy tablice dokumentów:
\begin{lstlisting}[language=JavaScript,caption=Przykład dokumentu w formacie JSON]
{
    firstname: "Jan",
    lastname: "Kowalski",
    age: 40
}
\end{lstlisting}


\section{Apache JMeter}
Apache JMeter(http://jmeter.apache.org/) jest programem służącym do wykonywania testów aplikacji w celu zmierzenia jej wydajności.  Początkowo został stworzony do tworzenia testów serwisów internetowych. Jednak z czasem został on rozszerzony o dodatkowe funkcje. Apache JMeter można użyć do symulowania wysokiego obciążenia aplikacji na serwerze, sieci lub innych testowanych obiektach.
Obecnie JMeter można zastosować do testowania serwerów i protokołów:
\begin{itemize}
\item HTTP
\item HTTPS
\item FTP
\item SOAP oraz REST
\item relacyjne bazy danych - przy użyciu sternika JDBC
\item nierelacyjne bazy danych np MongoDB
\item usług pocztowych wykorzstujących protokoły: SMTP, POP3 oraz IMAP
\item TCP
\end{itemize}
Apache JMeter można rozszerzać o własne pluginy więc lista usług dostępnych do testowania jest nieograniczona. \\*
Apache JMeter jest wielowątkowym narzędziem, przez co można wykonywać ten sam test rówlnolegle, symulując w ten sposób wielu urzytkowników. \\
% TODO Jak tworzyć testy
Pierwsza stabilna wersja Apache JMeter została wydana 15 grudnia 1998r. Dla testów w pracy użyto wersji 2.13, która została wydana 14 marca 2015r.

\section{Digitalocean}

\chapter{Projekt Aplikacji}
\section{Opis}
\subsection{Infrastruktura}
Do przeprowadzenia testów wydajnościowych wykorzystywane były serwery:
\begin{itemize}
    \item serwer aplikacji - serwer na którym uruchomiona była testowana aplikacja
    \item serwer bazy MongoDB
    \item serwer testowy z aplikacją \textsl{Apache JMeter} skąd prowadzone były testy
\end{itemize}
Serwer z aplikacją 

Wszystkie serwery zostały wykupione w serwerowni Frankfurt. Komunikacja odbywała się po sieci lokalnej \textsl{LAN}.

\section{Testy integracyjne}

\chapter{Testy wydajnościowe}
\section{Opis testów}

Do testów przygotowano dwie aplikację. Pierwszą z nich była aplikacja w języku \textsl{Java} uruchomiona na dwóch różnych serwerach: \textsl{Tomcat 8} i \textsl{Jetty 9}. Serwery te są obecnie jednymi z najpopularniejszych rozwiązań służących do uruchamiania aplikacji \textsl{Java}.  Drugą aplikacją była aplikacja w języku \textsl{GO}, która przy użyciu biblioteki \textsl{HttpRouter} (https://github.com/julienschmidt/httprouter) pozwala na tworzenie aplikacji działającej jako serwer \textsl{HTTP}.

Każda aplikacja była testowana przez przygotowany zbiór testów w aplikacji \textsl{Apache JMeter}. \textsl{Apache JMeter} pozwala na tworzenie i wykonywanie testów wydajnościowych, symulujących wielu klientów korzystających z aplikacji. Przygotowane testy zostały podzielone na 4 grupy:
\begin{itemize}
    \item testy sprawdzające wydajność walidacji istnienia klucza \textsl{API}
    \item testy sprawdzające wydajność walidacji obiektu \textsl{Cache}
    \item testy sprawdzające wydajność operacji typu \textsl{CRUD}
    \item testy sprawdzające wydajność powyższych scenariuszy równolegle
\end{itemize}

Pierwsza grupa sprawdzała wydajność aplikacji, gdy klient chciał wykonać operacje posiadając nieistniejący klucz \textsl{API}. Na tą grupę składało się pięć testów, wykonywanej w następującej kolejności:
\begin{enumerate}
    \item pobieranie listy wszyskitch obiektów \textsl{Cache}
    \item pobieranie pojedynczego obiektu \textsl{Cache}
    \item tworzenie obiektu \textsl{Cache}
    \item aktualizacji obiektu \textsl{Cache} 
    \item usunięcie obiektu \textsl{Cache}
\end{enumerate}
Każdy z tych testów oznaczany był jako poprawny, gdy aplikacja zwracała błąd autoryzacji dla każdego żądania.

Drugą grupą testów było sprawdzenie wydajności, gdy klucz \textsl{API} istniał jednak klient chciał przeprowadzić operacje pobierania, usuwania i aktualizacji nie istniejącego obiektu \textsl{Cache}. Scenariusz grupy wyglądał następująco:
\begin{enumerate}
    \item pobierz nowy klucz \textsl{API}
    \item pobierz obiekt \textsl{Cache} autoryzując się otrzymanym kluczem \textsl{API}
    \item zaktualizuj obiekt \textsl{Cache} autoryzując się otrzymanym kluczem \textsl{API}
    \item usuń obiekt \textsl{Cache} autoryzując się otrzymanym kluczem \textsl{API}
\end{enumerate}
Każdy z tych testów oznaczany był jako poprawny, gdy aplikacja zwracała błąd autoryzacji dla każdego żądania.

Kolejną grupą były testy wydajności operacji \textsl{CRUD}. Do przeprowadzenia tej grupy testów został przygotowany zbiór 100 tysięcy losowych wartości w formie pliku \textsl{CSV}. Każda wartość składała się 3 części. Pierwszą był klucz obiektu \textsl{Cache}, natomiast dwie kolejne to wartości, które zostaną zapisane w aplikacji w polu \textsl{value} obiektu. \textsl{Apache JMeter} pozwala na przekazanie pliku \textsl{CSV}, którego wartości można użyć do przeprowadzenia testów. Scenariusz tej grupy wyglądał następująco:
\begin{enumerate}
    \item pobierz nowy klucz \textsl{API}
    \item utwórz obiekt \textsl{Cache} o kluczu i wartości otrzymanym z parametru
    \item pobierz utworzony obiekt \textsl{Cache} 
    \item zaktualizuj obiekt \textsl{Cache} ustawiając nową wartość pola \textsl{value}
    \item usuń obiekt \textsl{Cache}
\end{enumerate}
Każdy z tych testów oznaczany był jako poprawny, gdy aplikacja dla każdego z nich nie zwracała błędu.

Ostatnią grupę stanowiły wszystkie testy, które zostały opisane w powyższych grupach. 

Wszystkie grupy były wykonywane w 15 minutowych cyklach, oddzielonych 60 sekundową przerwą.

Każda aplikacja testowana była w czterech przypadkach testowych. Przypadki te różniły się od siebie liczbą klientów, którzy równolegle wykonywali żądania oraz stanu początkowego bazy danych:
\begin{itemize}
    \item 100 klientów oraz pusta baza danych
    \item 250 klientów oraz pusta baza danych
    \item 100 klientów oraz baza danych wypełniona danymi
    \item 250 klientów oraz baza danych wypełniona danymi
\end{itemize}
W dwóch ostatnich przypadkach baza danych była wypełniona losowymi danymi: 4000000 obiektów w kolekcji \textsl{api}, 40000000 obiektów w kolekcji \textsl{cache}. Łącznie baza danych zajmowała 12 gigabajtów pamięci masowej.

\section{Środowisko testowe}
Do przeprowadzenia testów wydajnościowych wykorzystywane były 3 serwery wirtualne. Specyfikacje techniczne serwerów wirtualnych, wykorzystanych w testach to: 
\begin{itemize}
    \item 8 rdzeni, 16 gigabajtów pamięci RAM, 160 gigabajtów dysku SSD dla serwera aplikacyjnego
    \item 4 rdzenie, 8 gigabajtów pamięci RAM, 80 gigabajtów dysku SSD dla serwera bazy danych 
    \item 4 rdzenie, 8 gigabajtów pamięci RAM, 80 gigabajtów dysku SSD dla serwera, na którym uruchomiony był program \textsl{Apache JMeter}
\end{itemize}
Serwery komunikowały się po sieci lokalnej (ang. \textsl{LAN}) bezpośrednio między sobą.

Na rysunku \ref{fig:deployment_diagram} zaprezentowany został diagram wdrożenia infrastruktury wykorzystanej do przeprowadzenia testów.
\begin{figure}[!ht]
\centering
\includegraphics[width=12cm, height=9cm]{\ImgPath/diagram_wdrozenia.png}
\caption{Diagram wdrożenia infrastruktury wykorzystywanej do przeprowadzenia testów}
\label{fig:deployment_diagram}
\end{figure}

\newpage
\section{Wyniki testów - pusta baza danych}

\subsection{Testy wydajności walidacji API}
Pierwszą grupą testów były testy walidacji \textsl{API}. Z diagramów \ref{fig:tomcat_clean_api_validation_rps} i \ref{fig:tomcat_clean_api_validation_td} przedstawiających wydajność aplikacji \textsl{Java} uruchomionej na serwerze \textsl{Tomcat 8} wynika, że przy 100 równolegle uruchomionych klientach liczba obsłużonych żądań wahała się od 6 do 8 tysięcy żądań w ciągu sekundy. Natomiast średni czas odpowiedzi wynosił 11.82 milisekund. W przypadku 250 klientów, którzy testowali aplikację liczba żądań wahała się miedzy 4 a 8 tysiącami, a rozkład czasów odpowiedzi bardzo się spłaszczył. Średni czas odpowiedzi wynosił w tym teście 33.73 milisekundy.

Wyniki testów dla aplikacji \textsl{Java} uruchomionej na serwerze \textsl{Jetty 9} przedstawione są na diagramach \ref{fig:jetty_clean_api_validation_rps} i \ref{fig:jetty_clean_api_validation_td}. W teście przy 100 klientach liczba obsłużonych żądań w ciągu sekundy wahała się od 7 do 10 tysięcy, a średni czas odpowiedzi wyniósł 9.30 milisekund. Na diagramie rozkładu czasów odpowiedzi widać, że największa ilość żądań trwała poniżej 10 milisekund. Dla testu z uruchomionymi 250 klientami liczba żądań wahała się między 5 a 10 tysięcy i rozkład czasów również się spłaszczył. W teście tym średni czas odpowiedzi wyniósł 33.66 milisekund. 

Ostatnią testowaną aplikacją była aplikacja napisana w języku \textsl{Go}. Rozkład liczby żądań obsłużonych w ciągu sekundy (\ref{fig:go_clean_api_validation_rps}) dla testu z wykorzystaniem 100 i 250 klientów jest bardzo zbliżony. Dla 100 klientów oscyluje w okolicy 10 tysięcy żądań, dla 250 klientów w okolicy 9 tysięcy żądań. Analizując rozkład czasów odpowiedzi (\ref{fig:go_clean_api_validation_td}) można zauważyć, że w obu testach jest on symetryczny. Średnie czasy odpowiedzi wyniosły: 8.31 milisekund dla 100 klientów oraz 23.80 milisekund dla 250 klientów.

W tabelach \ref{tab:app-clean-api} oraz \ref{tab:mongo-clean-api} zaprezentowane są średnie wykorzystanie procesora oraz pamięci \textsl{RAM} na maszynach, gdzie uruchomione były aplikacje oraz baza danych \textsl{MongoD}. Aplkacja \textsl{Java} uruchomiona na serwerze \textsl{Tomcat 8} wykorzystywała najbardziej zasoby procesora (77.50\% i 65.73\%), serwer \textsl{Jetty 9} obciążał o około 5\% mniej procesor. Natomiast aplikacja w języku \textsl{Go} wymagała najmniej zasobów procesora (29.93\% i 31.54\%). Serwer \textsl{Jetty 9} potrzebował najwięcej pamięci \textsl{RAM} do działania (ponad 2000 MB), \textsl{Tomcat 8} potrzebował o 100 MB mniej, a aplikacja w języku \textsl{Go} potrzebowała około 210 MB pamięci. 

% \pgfplotsset{grid style={dashed}}
\begin{figure}[!ht]
\pgfplotstableread[col sep = comma]{csv_queries/requests_per_sec/tomcat_clean_api_validation.csv}\csvdata
\begin{tikzpicture}
  \begin{axis}[xmin = 0, xmax=900, ymin = 0, scaled y ticks = base 10:-3, xlabel = {Czas [s]}, ylabel = Liczba żądań, legend pos=south east, ymajorgrids,width=13cm, height=6cm] %TODO miary?
    \addplot[color=blue,mark=none] table[x index=0, y index=1]{\csvdata};
    \addplot[color=green,mark=none] table[x index=0, y index=2]{\csvdata};
    \legend{100,250}
  \end{axis}
\end{tikzpicture}
\caption{Tomcat 8 - liczba żądań obsłużonych przez aplikację w ciągu sekundy podczas testu walidacji istnienia klucza API}
\label{fig:tomcat_clean_api_validation_rps}
\end{figure}

\begin{figure}[!ht]
\pgfplotstableread[col sep = comma]{csv_queries/response_time_distribution/tomcat_clean_api_validation_100.csv}\csva
\pgfplotstableread[col sep = comma]{csv_queries/response_time_distribution/tomcat_clean_api_validation_250.csv}\csvb
\pgfplotsset{
    /pgfplots/ybar legend/.style={
    /pgfplots/legend image code/.code={\draw[##1,/tikz/.cd,yshift=-0.25em](0cm,0cm) rectangle(1pt,0.7em);},
   }
}
\begin{tikzpicture}
  \begin{axis}[ybar, bar width=0.5, xmin = 0, ymin = 0, scaled y ticks = base 10:-5, xlabel = {Czas odpowiedzi [ms]}, ylabel = {Liczba żądań}, ymajorgrids,width=13cm, height=6cm] %TODO miary?
    \addplot[color=blue, mark=none, fill=blue] table[x index=0, y index=1]{\csva};
    \addplot[color=green, mark=none, fill=green] table[x index=0, y index=1]{\csvb};
    \legend{100,250}
  \end{axis}
\end{tikzpicture}
\caption{Tomcat 8 - rozkład czasów odpowiedzi aplikacji (95\% odpowiedzi) podczas testu walidacji istnienia klucza API}
\label{fig:tomcat_clean_api_validation_td}
\end{figure}

\pgfplotsset{grid style={dashed}}
\begin{figure}[!ht]
\pgfplotstableread[col sep = comma]{csv_queries/requests_per_sec/jetty_clean_api_validation.csv}\csvdata
\begin{tikzpicture}
  \begin{axis}[xmin = 0, xmax=900, ymin = 0, scaled y ticks = base 10:-3, xlabel = {Czas [s]}, ylabel = Liczba żądań, legend pos=south east, ymajorgrids,width=13cm, height=6cm] %TODO miary?
    \addplot[color=blue,mark=none] table[x index=0, y index=1]{\csvdata};
    \addplot[color=green,mark=none] table[x index=0, y index=2]{\csvdata};
    \legend{100,250}
  \end{axis}
\end{tikzpicture}
\caption{Jetty 9 - liczba żądań obsłużonych przez aplikację w ciągu sekundy podczas testu walidacji istnienia klucza API}
\label{fig:jetty_clean_api_validation_rps}
\end{figure}

\begin{figure}[!ht]
\pgfplotstableread[col sep = comma]{csv_queries/response_time_distribution/jetty_clean_api_validation_100.csv}\csva
\pgfplotstableread[col sep = comma]{csv_queries/response_time_distribution/jetty_clean_api_validation_250.csv}\csvb
\pgfplotsset{
    /pgfplots/ybar legend/.style={
    /pgfplots/legend image code/.code={\draw[##1,/tikz/.cd,yshift=-0.25em](0cm,0cm) rectangle(1pt,0.7em);},
   }
}
\begin{tikzpicture}
  \begin{axis}[ybar, bar width=0.5, xmin = 0, ymin = 0, scaled y ticks = base 10:-5, xlabel = {Czas odpowiedzi [ms]}, ylabel = {Liczba żądań}, ymajorgrids,width=13cm, height=6cm] %TODO miary?
    \addplot[color=blue, mark=none, fill=blue] table[x index=0, y index=1]{\csva};
    \addplot[color=green, mark=none, fill=green] table[x index=0, y index=1]{\csvb};
    \legend{100,250}
  \end{axis}
\end{tikzpicture}
\caption{Jetty 9 - rozkład czasów odpowiedzi aplikacji (95\% odpowiedzi) podczas testu walidacji istnienia klucza API}
\label{fig:jetty_clean_api_validation_td}
\end{figure}

\pgfplotsset{grid style={dashed}}
\begin{figure}[!ht]
\pgfplotstableread[col sep = comma]{csv_queries/requests_per_sec/go_clean_api_validation.csv}\csvdata
\begin{tikzpicture}
  \begin{axis}[xmin = 0, xmax=900, ymin = 0, scaled y ticks = base 10:-3, xlabel = {Czas [s]}, ylabel = Liczba żądań, legend pos=south east, ymajorgrids,width=13cm, height=6cm] %TODO miary?
    \addplot[color=blue,mark=none] table[x index=0, y index=1]{\csvdata};
    \addplot[color=green,mark=none] table[x index=0, y index=2]{\csvdata};
    \legend{100,250}
  \end{axis}
\end{tikzpicture}
\caption{Go - liczba żądań obsłużonych przez aplikację w ciągu sekundy podczas testu walidacji istnienia klucza API}
\label{fig:go_clean_api_validation_rps}
\end{figure}

\begin{figure}[!ht]
\pgfplotstableread[col sep = comma]{csv_queries/response_time_distribution/go_clean_api_validation_100.csv}\csva
\pgfplotstableread[col sep = comma]{csv_queries/response_time_distribution/go_clean_api_validation_250.csv}\csvb
\pgfplotsset{
    /pgfplots/ybar legend/.style={
    /pgfplots/legend image code/.code={\draw[##1,/tikz/.cd,yshift=-0.25em](0cm,0cm) rectangle(1pt,0.7em);},
   }
}
\begin{tikzpicture}
  \begin{axis}[ybar, bar width=0.5, xmin = 0, ymin = 0, scaled y ticks = base 10:-5, xlabel = {Czas odpowiedzi [ms]}, ylabel = {Liczba żądań}, ymajorgrids,width=13cm, height=6cm] %TODO miary?
    \addplot[color=blue, mark=none, fill=blue] table[x index=0, y index=1]{\csva};
    \addplot[color=green, mark=none, fill=green] table[x index=0, y index=1]{\csvb};
    \legend{100,250}
  \end{axis}
\end{tikzpicture}
\caption{Go - rozkład czasów odpowiedzi aplikacji (95\% odpowiedzi) podczas testu walidacji istnienia klucza API}
\label{fig:go_clean_api_validation_td}
\end{figure}



\begin{table}[!htb]
\centering
\caption{Wykorzystanie procesora i pamięci RAM na serwerze, gdzie uruchomiona była aplikacja}
\label{tab:app-clean-api}
\begin{tabular}{@{}ccccl@{}}
\toprule
\textbf{Język} & \textbf{Liczba wątków} & \multicolumn{1}{p{3cm}}{\textbf{Średnie wykorzystanie CPU (\%)}} & \multicolumn{1}{p{3cm}}{\textbf{Średnie wykorzystanie RAM (MB)}} &  \\ \midrule
Tomcat 8       & 100                    & 77.5                             & 1882.6                          &  \\
Tomcat 8       & 250                    & 65.73                             & 1926.08                          &  \\
Jetty 9       & 100                    & 70.13                             & 2020.03                          &  \\
Jetty 9       & 250                    & 59.31                             & 2005.06                          &  \\
Go       & 100                    & 29.93                             & 211.65                          &  \\
Go       & 250                    & 31.64                             & 228.72                          &  \\
\bottomrule
\end{tabular}
\end{table}


\begin{table}[!htb]
\centering
\caption{Wykorzystanie procesora i pamięci RAM na serwerze, gdzie uruchomiona była baza danych MongoDB}
\label{tab:mongo-clean-api}
\begin{tabular}{@{}ccccl@{}}
\toprule
\textbf{Język} & \textbf{Liczba wątków} & \multicolumn{1}{p{3cm}}{\textbf{Średnie wykorzystanie CPU (\%)}} & \multicolumn{1}{p{3cm}}{\textbf{Średnie wykorzystanie RAM (MB)}} &  \\ \midrule
Tomcat 8       & 100                    & 29.4                             & 151.1                          &  \\
Tomcat 8       & 250                    & 22.05                             & 272.9                          &  \\
Jetty 9       & 100                    & 30.29                             & 301.42                          &  \\
Jetty 9       & 250                    & 22.55                             & 270.23                          &  \\
Go       & 100                    & 18.92                             & 302.67                          &  \\
Go       & 250                    & 16.84                             & 358.49                          &  \\
\bottomrule
\end{tabular}
\end{table}


\clearpage

\subsection{Test wydajności walidacji kluczy w bazie}
% \input{chapters/5_testy_wydajnosciowe_diagram_2_clean_key_validation.tex}

\begin{table}[!htb]
\centering
\caption{Wykorzystanie procesora i pamięci RAM na serwerze, gdzie uruchomiona była aplikacja}
\label{tab:app-clean-key}
\begin{tabular}{@{}ccccl@{}}
\toprule
\textbf{Język} & \textbf{Liczba wątków} & \multicolumn{1}{p{3cm}}{\textbf{Średnie wykorzystanie CPU (\%)}} & \multicolumn{1}{p{3cm}}{\textbf{Średnie wykorzystanie RAM (MB)}} &  \\ \midrule
Tomcat 8       & 100                    & 50.63                             & 1935.12                          &  \\
Tomcat 8       & 250                    & 49.02                             & 2009.8                          &  \\
Jetty 9       & 100                    & 50.61                             & 1949.84                          &  \\
Jetty 9       & 250                    & 45.24                             & 1851.88                          &  \\
Go       & 100                    & 39.28                             & 214.6                          &  \\
Go       & 250                    & 45.59                             & 234.03                          &  \\
\bottomrule
\end{tabular}
\end{table}


\begin{table}[!htb]
\centering
\caption{Wykorzystanie procesora i pamięci RAM na serwerze, gdzie uruchomiona była baza danych MongoDB}
\label{tab:mongo-clean-key}
\begin{tabular}{@{}ccccl@{}}
\toprule
\textbf{Język} & \textbf{Liczba wątków} & \multicolumn{1}{p{3cm}}{\textbf{Średnie wykorzystanie CPU (\%)}} & \multicolumn{1}{p{3cm}}{\textbf{Średnie wykorzystanie RAM (MB)}} &  \\ \midrule
Tomcat 8       & 100                    & 44.74                             & 167.05                          &  \\
Tomcat 8       & 250                    & 39.64                             & 277.01                          &  \\
Jetty 9       & 100                    & 58.65                             & 260.65                          &  \\
Jetty 9       & 250                    & 43.34                             & 268.43                          &  \\
Go       & 100                    & 42.72                             & 302.94                          &  \\
Go       & 250                    & 41.54                             & 365.98                          &  \\
\bottomrule
\end{tabular}
\end{table}


\clearpage

\subsection{Test wydajności operacji CRUD}
% \input{chapters/5_testy_wydajnosciowe_diagram_3_clean_crud.tex}

\begin{table}[!htb]
\centering
\caption{Wykorzystanie procesora i pamięci RAM na serwerze, gdzie uruchomiona była aplikacja}
\label{tab:app-clean-crud}
\begin{tabular}{@{}ccccl@{}}
\toprule
\textbf{Język} & \textbf{Liczba wątków} & \multicolumn{1}{p{3cm}}{\textbf{Średnie wykorzystanie CPU (\%)}} & \multicolumn{1}{p{3cm}}{\textbf{Średnie wykorzystanie RAM (MB)}} &  \\ \midrule
Tomcat 8       & 100                    & 42.38                             & 1959.11                          &  \\
Tomcat 8       & 250                    & 40.25                             & 2034.83                          &  \\
Jetty 9       & 100                    & 36.32                             & 1753.27                          &  \\
Jetty 9       & 250                    & 39.60                             & 1605.56                          &  \\
Go       & 100                    & 33.31                             & 222.13                          &  \\
Go       & 250                    & 41.36                             & 242.23                          &  \\
\bottomrule
\end{tabular}
\end{table}


\begin{table}[!htb]
\centering
\caption{Wykorzystanie procesora i pamięci RAM na serwerze, gdzie uruchomiona była baza danych MongoDB}
\label{tab:mongo-clean-crud}
\begin{tabular}{@{}ccccl@{}}
\toprule
\textbf{Język} & \textbf{Liczba wątków} & \multicolumn{1}{p{3cm}}{\textbf{Średnie wykorzystanie CPU (\%)}} & \multicolumn{1}{p{3cm}}{\textbf{Średnie wykorzystanie RAM (MB)}} &  \\ \midrule
Tomcat 8       & 100                    & 59.35                             & 174.04                          &  \\
Tomcat 8       & 250                    & 53.36                             & 276.73                          &  \\
Jetty 9       & 100                    & 59.18                             & 247.87                          &  \\
Jetty 9       & 250                    & 55.55                             & 269.80                          &  \\
Go       & 100                    & 40.46                             & 303.32                          &  \\
Go       & 250                    & 39.74                             & 361.53                          &  \\
\bottomrule
\end{tabular}
\end{table}


\clearpage

\subsection{Test wydajności walidacji API, walidacji kluczy w bazie oraz operacji CRUD rówlolegle }
% \input{chapters/5_testy_wydajnosciowe_diagram_4_clean_all.tex}

\begin{table}[!htb]
\centering
\caption{Wykorzystanie procesora i pamięci RAM na serwerze, gdzie uruchomiona była aplikacja}
\label{tab:app-clean-all}
\begin{tabular}{@{}ccccl@{}}
\toprule
\textbf{Język} & \textbf{Liczba wątków} & \multicolumn{1}{p{3cm}}{\textbf{Średnie wykorzystanie CPU (\%)}} & \multicolumn{1}{p{3cm}}{\textbf{Średnie wykorzystanie RAM (MB)}} &  \\ \midrule
Tomcat 8       & 100                    & 56.12                             & 1976.99                          &  \\
Tomcat 8       & 250                    & 46.62                             & 2063.85                          &  \\
Jetty 9       & 100                    & 51.55                             & 1944.2                          &  \\
Jetty 9       & 250                    & 46.04                             & 1600.48                          &  \\
Go       & 100                    & 40.59                             & 223.55                          &  \\
Go       & 250                    & 46.66                             & 244.16                          &  \\
\bottomrule
\end{tabular}
\end{table}


\begin{table}[!htb]
\centering
\caption{Wykorzystanie procesora i pamięci RAM na serwerze, gdzie uruchomiona była baza danych MongoDB}
\label{tab:mongo-clean-all}
\begin{tabular}{@{}ccccl@{}}
\toprule
\textbf{Język} & \textbf{Liczba wątków} & \multicolumn{1}{p{3cm}}{\textbf{Średnie wykorzystanie CPU (\%)}} & \multicolumn{1}{p{3cm}}{\textbf{Średnie wykorzystanie RAM (MB)}} &  \\ \midrule
Tomcat 8       & 100                    & 55                             & 175.78                          &  \\
Tomcat 8       & 250                    & 42.89                             & 275.74                          &  \\
Jetty 9       & 100                    & 61.82                             & 248.13                          &  \\
Jetty 9       & 250                    & 48.07                             & 269.2                          &  \\
Go       & 100                    & 41.66                             & 305.38                          &  \\
Go       & 250                    & 39.77                             & 359.53                          &  \\
\bottomrule
\end{tabular}
\end{table}


\clearpage

\newpage
\section{Wyniki testów - baza danych wypełniona danymi początkowymi}
\subsection{Testy wydajnośsci walidacji API}
% \input{chapters/5_testy_wydajnosciowe_diagram_5_full_api_validation.tex}

\begin{table}[!htb]
\centering
\caption{Wykorzystanie procesora i pamięci RAM na serwerze, gdzie uruchomiona była aplikacja}
\label{tab:app-full-api}
\begin{tabular}{@{}ccccl@{}}
\toprule
\textbf{Język} & \textbf{Liczba wątków} & \multicolumn{1}{p{3cm}}{\textbf{Średnie wykorzystanie CPU (\%)}} & \multicolumn{1}{p{3cm}}{\textbf{Średnie wykorzystanie RAM (MB)}} &  \\ \midrule
Tomcat 8       & 100                    & 74.01                             & 1932.11                          &  \\
Tomcat 8       & 250                    & 47.96                             & 1965.24                          &  \\
Jetty 9       & 100                    & 81.33                             & 1990.89                          &  \\
Jetty 9       & 250                    & 59.42                             & 1980.65                          &  \\
Go       & 100                    & 31.35                             & 241.75                          &  \\
Go       & 250                    & 32.38                             & 249.13                          &  \\
\bottomrule
\end{tabular}
\end{table}


\begin{table}[!htb]
\centering
\caption{Wykorzystanie procesora i pamięci RAM na serwerze, gdzie uruchomiona była baza danych MongoDB}
\label{tab:mongo-full-api}
\begin{tabular}{@{}ccccl@{}}
\toprule
\textbf{Język} & \textbf{Liczba wątków} & \multicolumn{1}{p{3cm}}{\textbf{Średnie wykorzystanie CPU (\%)}} & \multicolumn{1}{p{3cm}}{\textbf{Średnie wykorzystanie RAM (MB)}} &  \\ \midrule
Tomcat 8       & 100                    & 24.54                             & 464.09                          &  \\
Tomcat 8       & 250                    & 14.05                             & 463.10                         &  \\
Jetty 9       & 100                    & 33.73                             & 488.00                          &  \\
Jetty 9       & 250                    & 17.44                             & 464.01                          &  \\
Go       & 100                    & 18.66                             & 141.83                          &  \\
Go       & 250                    & 19.16                             & 144.53                          &  \\
\bottomrule
\end{tabular}
\end{table}


\clearpage

\subsection{Test wydajności walidacji kluczy w bazie}
% \input{chapters/5_testy_wydajnosciowe_diagram_6_full_key_validation.tex}

\begin{table}[!htb]
\centering
\caption{Wykorzystanie procesora i pamięci RAM na serwerze, gdzie uruchomiona była aplikacja}
\label{tab:app-full-key}
\begin{tabular}{@{}ccccl@{}}
\toprule
\textbf{Język} & \textbf{Liczba wątków} & \multicolumn{1}{p{3cm}}{\textbf{Średnie wykorzystanie CPU (\%)}} & \multicolumn{1}{p{3cm}}{\textbf{Średnie wykorzystanie RAM (MB)}} &  \\ \midrule
Tomcat 8       & 100                    & 44.94                             & 1996.10                          &  \\
Tomcat 8       & 250                    & 39.30                            & 2040.71                          &  \\
Jetty 9       & 100                    & 51.93                             & 2051.50                          &  \\
Jetty 9       & 250                    & 42.63                             & 2049.01                          &  \\
Go       & 100                    & 29.86                             & 244.53                          &  \\
Go       & 250                    & 34.68                             & 251.53                          &  \\
\bottomrule
\end{tabular}
\end{table}


\begin{table}[!htb]
\centering
\caption{Wykorzystanie procesora i pamięci RAM na serwerze, gdzie uruchomiona była baza danych MongoDB}
\label{tab:mongo-full-key}
\begin{tabular}{@{}ccccl@{}}
\toprule
\textbf{Język} & \textbf{Liczba wątków} & \multicolumn{1}{p{3cm}}{\textbf{Średnie wykorzystanie CPU (\%)}} & \multicolumn{1}{p{3cm}}{\textbf{Średnie wykorzystanie RAM (MB)}} &  \\ \midrule
Tomcat 8       & 100                    & 42.32                             & 435.73                          &  \\
Tomcat 8       & 250                    & 43.57                             & 458.27                          &  \\
Jetty 9       & 100                    & 60.13                             & 437.91                          &  \\
Jetty 9       & 250                    & 40.57                             & 436.32                          &  \\
Go       & 100                    & 38.89                             & 181.49                          &  \\
Go       & 250                    & 41.61                             & 189.06                          &  \\
\bottomrule
\end{tabular}
\end{table}


\clearpage

\subsection{Test wydajności operacji CRUD}
% \input{chapters/5_testy_wydajnosciowe_diagram_7_full_crud.tex}

\begin{table}[!htb]
\centering
\caption{Wykorzystanie procesora i pamięci RAM na serwerze, gdzie uruchomiona była aplikacja}
\label{tab:app-full-crud}
\begin{tabular}{@{}ccccl@{}}
\toprule
\textbf{Język} & \textbf{Liczba wątków} & \multicolumn{1}{p{3cm}}{\textbf{Średnie wykorzystanie CPU (\%)}} & \multicolumn{1}{p{3cm}}{\textbf{Średnie wykorzystanie RAM (MB)}} &  \\ \midrule
Tomcat 8       & 100                    & 28.86                             & 1752.04                          &  \\
Tomcat 8       & 250                    & 27.06                             & 2053.7                          &  \\
Jetty 9       & 100                    & 27.3                             & 1951.18                          &  \\
Jetty 9       & 250                    & 28.24                             & 2081.26                          &  \\
Go       & 100                    & 23.9                             & 251.78                          &  \\
Go       & 250                    & 23.23                             & 260.53                          &  \\
\bottomrule
\end{tabular}
\end{table}


\begin{table}[!htb]
\centering
\caption{Wykorzystanie procesora i pamięci RAM na serwerze, gdzie uruchomiona była baza danych MongoDB}
\label{tab:mongo-full-crud}
\begin{tabular}{@{}ccccl@{}}
\toprule
\textbf{Język} & \textbf{Liczba wątków} & \multicolumn{1}{p{3cm}}{\textbf{Średnie wykorzystanie CPU (\%)}} & \multicolumn{1}{p{3cm}}{\textbf{Średnie wykorzystanie RAM (MB)}} &  \\ \midrule
Tomcat 8       & 100                    & 54.46                             & 438.29                          &  \\
Tomcat 8       & 250                    & 53.02                             & 449.68                          &  \\
Jetty 9       & 100                    & 52.66                             & 441.22                          &  \\
Jetty 9       & 250                    & 51.48                             & 442.63                          &  \\
Go       & 100                    & 37.04                             & 256.85                          &  \\
Go       & 250                    & 37.27                             & 264.31                          &  \\
\bottomrule
\end{tabular}
\end{table}


\clearpage

\subsection{Test wydajności walidacji API, walidacji kluczy w bazie oraz operacji CRUD równolegle }
% \pgfplotsset{grid style={dashed}}
\begin{figure}[!ht]
\pgfplotstableread[col sep = comma]{csv_queries/requests_per_sec/tomcat_full_all.csv}\csvdata
\begin{tikzpicture}
  \begin{axis}[xmin = 0, xmax=900, ymin = 0, scaled y ticks = base 10:-3, xlabel = {Czas [s]}, ylabel = Liczba żądań, legend pos=south east, ymajorgrids] %TODO miary?
    \addplot[color=blue,mark=none] table[x index=0, y index=1]{\csvdata};
    \addplot[color=green,mark=none] table[x index=0, y index=2]{\csvdata};
    \legend{100,250}
  \end{axis}
\end{tikzpicture}
\caption{Tomcat 8 - liczba żądań obsłużonych przez aplikację w ciągu sekundy podczas testu: walidacji istnienia klucza API, walidacji istnienia, operacji CRUD równolegle}
\label{fig:tomcat_full_all_rps}
\end{figure}

\begin{figure}[!ht]
\pgfplotstableread[col sep = comma]{csv_queries/response_time_distribution/tomcat_full_all_100.csv}\csva
\pgfplotstableread[col sep = comma]{csv_queries/response_time_distribution/tomcat_full_all_250.csv}\csvb
\pgfplotsset{
    /pgfplots/ybar legend/.style={
    /pgfplots/legend image code/.code={\draw[##1,/tikz/.cd,yshift=-0.25em](0cm,0cm) rectangle(1pt,0.7em);},
   }
}
\begin{tikzpicture}
  \begin{axis}[ybar, bar width=0.5, xmin = 0, ymin = 0, scaled y ticks = base 10:-5, xlabel = {Czas odpowiedzi [ms]}, ylabel = {Liczba żądań}, ymajorgrids] %TODO miary?
    \addplot[color=blue, mark=none, fill=blue] table[x index=0, y index=1]{\csva};
    \addplot[color=green, mark=none, fill=green] table[x index=0, y index=1]{\csvb};
    \legend{100,250}
  \end{axis}
\end{tikzpicture}
\caption{Tomcat 8 - rozkład czasów odpowiedzi aplikacji (95\% odpowiedzi) podczas testu: walidacji istnienia klucza API, walidacji istnienia, operacji CRUD równolegle}
\label{fig:tomcat_full_all_td}
\end{figure}

\pgfplotsset{grid style={dashed}}
\begin{figure}[!ht]
\pgfplotstableread[col sep = comma]{csv_queries/requests_per_sec/jetty_full_all.csv}\csvdata
\begin{tikzpicture}
  \begin{axis}[xmin = 0, xmax=900, ymin = 0, scaled y ticks = base 10:-3, xlabel = {Czas [s]}, ylabel = Liczba żądań, legend pos=south east, ymajorgrids] %TODO miary?
    \addplot[color=blue,mark=none] table[x index=0, y index=1]{\csvdata};
    \addplot[color=green,mark=none] table[x index=0, y index=2]{\csvdata};
    \legend{100,250}
  \end{axis}
\end{tikzpicture}
\caption{Jetty 9 - liczba żądań obsłużonych przez aplikację w ciągu sekundy podczas testu: walidacji istnienia klucza API, walidacji istnienia, operacji CRUD równolegle}
\label{fig:jetty_full_all_rps}
\end{figure}

\begin{figure}[!ht]
\pgfplotstableread[col sep = comma]{csv_queries/response_time_distribution/jetty_full_all_100.csv}\csva
\pgfplotstableread[col sep = comma]{csv_queries/response_time_distribution/jetty_full_all_250.csv}\csvb
\pgfplotsset{
    /pgfplots/ybar legend/.style={
    /pgfplots/legend image code/.code={\draw[##1,/tikz/.cd,yshift=-0.25em](0cm,0cm) rectangle(1pt,0.7em);},
   }
}
\begin{tikzpicture}
  \begin{axis}[ybar, bar width=0.5, xmin = 0, ymin = 0, scaled y ticks = base 10:-5, xlabel = {Czas odpowiedzi [ms]}, ylabel = {Liczba żądań}, ymajorgrids] %TODO miary?
    \addplot[color=blue, mark=none, fill=blue] table[x index=0, y index=1]{\csva};
    \addplot[color=green, mark=none, fill=green] table[x index=0, y index=1]{\csvb};
    \legend{100,250}
  \end{axis}
\end{tikzpicture}
\caption{Jetty 9 - rozkład czasów odpowiedzi aplikacji (95\% odpowiedzi) podczas testu: walidacji istnienia klucza API, walidacji istnienia, operacji CRUD równolegle}
\label{fig:jetty_full_all_td}
\end{figure}

\pgfplotsset{grid style={dashed}}
\begin{figure}[!ht]
\pgfplotstableread[col sep = comma]{csv_queries/requests_per_sec/go_full_all.csv}\csvdata
\begin{tikzpicture}
  \begin{axis}[xmin = 0, xmax=900, ymin = 0, scaled y ticks = base 10:-3, xlabel = {Czas [s]}, ylabel = Liczba żądań, legend pos=south east, ymajorgrids] %TODO miary?
    \addplot[color=blue,mark=none] table[x index=0, y index=1]{\csvdata};
    \addplot[color=green,mark=none] table[x index=0, y index=2]{\csvdata};
    \legend{100,250}
  \end{axis}
\end{tikzpicture}
\caption{Go - liczba żądań obsłużonych przez aplikację w ciągu sekundy podczas testu: walidacji istnienia klucza API, walidacji istnienia, operacji CRUD równolegle}
\label{fig:go_full_all_rps}
\end{figure}

\begin{figure}[!ht]
\pgfplotstableread[col sep = comma]{csv_queries/response_time_distribution/go_full_all_100.csv}\csva
\pgfplotstableread[col sep = comma]{csv_queries/response_time_distribution/go_full_all_250.csv}\csvb
\pgfplotsset{
    /pgfplots/ybar legend/.style={
    /pgfplots/legend image code/.code={\draw[##1,/tikz/.cd,yshift=-0.25em](0cm,0cm) rectangle(1pt,0.7em);},
   }
}
\begin{tikzpicture}
  \begin{axis}[ybar, bar width=0.5, xmin = 0, ymin = 0, scaled y ticks = base 10:-5, xlabel = {Czas odpowiedzi [ms]}, ylabel = {Liczba żądań}, ymajorgrids] %TODO miary?
    \addplot[color=blue, mark=none, fill=blue] table[x index=0, y index=1]{\csva};
    \addplot[color=green, mark=none, fill=green] table[x index=0, y index=1]{\csvb};
    \legend{100,250}
  \end{axis}
\end{tikzpicture}
\caption{Go - rozkład czasów odpowiedzi aplikacji (95\% odpowiedzi) podczas testu: walidacji istnienia klucza API, walidacji istnienia, operacji CRUD równolegle}
\label{fig:go_full_all_td}
\end{figure}



\begin{table}[!htb]
\centering
\caption{Wykorzystanie procesora i pamięci RAM na serwerze, gdzie uruchomiona była aplikacja}
\label{tab:app-full-all}
\begin{tabular}{@{}ccccl@{}}
\toprule
\textbf{Język} & \textbf{Liczba wątków} & \multicolumn{1}{p{3cm}}{\textbf{Średnie wykorzystanie CPU (\%)}} & \multicolumn{1}{p{3cm}}{\textbf{Średnie wykorzystanie RAM (MB)}} &  \\ \midrule
Tomcat 8       & 100                    & 43.92                             & 1641                          &  \\
Tomcat 8       & 250                    & 38.52                             & 2064.93                          &  \\
Jetty 9       & 100                    & 37.57                             & 1979.89                          &  \\
Jetty 9       & 250                    & 33.75                             & 2031.93                          &  \\
Go       & 100                    & 28.85                             & 253.25                          &  \\
Go       & 250                    & 30.64                             & 260.94                          &  \\
\bottomrule
\end{tabular}
\end{table}


\begin{table}[!htb]
\centering
\caption{Wykorzystanie procesora i pamięci RAM na serwerze, gdzie uruchomiona była baza danych MongoDB}
\label{tab:mongo-full-all}
\begin{tabular}{@{}ccccl@{}}
\toprule
\textbf{Język} & \textbf{Liczba wątków} & \multicolumn{1}{p{3cm}}{\textbf{Średnie wykorzystanie CPU (\%)}} & \multicolumn{1}{p{3cm}}{\textbf{Średnie wykorzystanie RAM (MB)}} &  \\ \midrule
Tomcat 8       & 100                    & 57.45                             & 463.62                          &  \\
Tomcat 8       & 250                    & 49.17                             & 466.37                          &  \\
Jetty 9       & 100                    & 58.51                             & 472.62                          &  \\
Jetty 9       & 250                    & 49.68                             & 469.12                          &  \\
Go       & 100                    & 40.57                             & 309.93                          &  \\
Go       & 250                    & 41.41                             & 314.7                          &  \\
\bottomrule
\end{tabular}
\end{table}


\clearpage

\newpage
\section{Podsumowanie wyników}

\input{chapters/6_wnioski.tex}
\appendix
\chapter{Implementacja serwisu języku Java}
\chapter{Implementacja serwisu języku Go}
\chapter{Testy integracyjne}

\setstretch{1.0}
\begin{thebibliography}{99}
\addcontentsline{toc}{chapter}{Bibliografia}
\bibitem{restinpractice}{Jim Webber, Savas Parastatidis, Ian Robinson, ''REST in Practice: Hypermedia and Systems Architecture", O'Reilly Media, 2010}
\bibitem{newman}{Sam Newman, "Building Microservices", O'Reilly Media, 2015.}
\bibitem{ddd}{Eric Evans, "Domain-Driven Design: Tackling Complexity in the Heart of Software", Addison-Wesley, 2003.}
\bibitem{java8}{Cay S. Horstmann, Gary Cornell, "Java. Podstawy. Wydanie IX", Helion, 2013.}
\bibitem{jeedesign}{Rod Jhnson, "Expert One-on-One J2EE Design and Development. New edition". Wrox, 2002.}
\bibitem{springinaction}{Craig Walls, "Spring w akcji. Wydanie III". Helion, 2013.}
\bibitem{programmingingo}{Mark Summerfield, "Programming in Go", Addison-Wesley Professional, 2012}
\bibitem{gophrasebook}{David Chisnall, "The Go Programming Language Phrasebook",  Addison-Wesley Professional, 2012}
\end{thebibliography}

% \zakonczenie  % wklejenie recenzji i opinii

\end{document}
%+++ END +++
