Język \textsl{Go}, potocznie nazywany \textsl{golang}, powstał w 2007 roku w firmie Google. Autorami jego byli Ken Thompson, Robert Griesemer oraz Rob Pike \cite{programmingingo}. Ponieważ każdy z nich miał swój wkład w rozwój języka \textsl{C}, chcieli  stworzyć język programowania, który będzie w momencie tworzenia tym, czym \textsl{C} był w latach 80. Autorzy języka chcieli, by nowy język pozwalał na tworzenie dużych systemów, które  łatwo będą się skalować, kompilacja ich będzie bardzo krótka, będzie statycznie typowany, a zarządzanie pamięcią nie będzie zmorą programistów. Przykład \ref{lst:gohelloworld} przedstawia program, który wypisuje tekst \textsl{Hello World}.
\begin{lstlisting}[language=Go, caption={Przykład programu w języku Go}, label={lst:gohelloworld}]
package main

import "fmt"

func main() {
	fmt.Println("Hello World")
}
\end{lstlisting}

\textsl{Go} był tworzony z myślą o programowaniu równoległym. Projektując, autorzy wzorowali się językami \textsl{Erlang} oraz \textsl{C}. W \textsl{Erlang} tworzenie programów wielowątkowych było proste i pozwalało na wykorzystanie wszystkich rdzeni systemu równomiernie. Wadą  \textsl{Erlang} była wydajność programów na systemach jednowątkowych. Język \textsl{C} był w tym względzie znacznie wydajniejszy, natomiast tworzenie programów operujących na kilku wątkach wymaga zastosowania bibliotek takich jak \textsl{POSIX} oraz \textsl{OpenMP}. W języku \textsl{C} pisanie skalowalnych aplikacji wymagało dodatkowej pracy.

\textsl{Go} łączy najlepsze cechy z obu języków, a do tworzenia systemów wielowątkowych wykorzystuje \textsl{go rutyny} (ang. \textsl{gorutine}) \cite{gophrasebook}. Są to funkcje o równoległym wykonywaniu. Wywoływana funkcja musi być poprzedzona słowem kluczowym \textsl{go}. Z wykorzystaniem kanałów programista jest w stanie zaimplementować komunikację między uruchomionymi funkcjami. Jednak wywołanie funkcji w \textsl{goroutine} nie oznacza stworzenia nowego wątku. 

W przykładzie \ref{lst:gogorutine} przedstawiony jest program uruchamiający funkcję wewnątrz \textsl{go rutyny}

\begin{lstlisting}[language=Go, caption={Przykładowy program w Go wykorzystujący goroutine}, label={lst:gogorutine}]
package main

import "fmt"

func say(s string) {
	fmt.Println(s)
}

func main() {
	go say("Hello World")
}
\end{lstlisting}

Kolejną rzeczą, którą autorzy języka postawili sobie na celu była jego szybka kompilacja. Programy w językach \textsl{C}/\textsl{C++} kompilowały się długo. Szybkość kompilacji \textsl{Go} pozwala nawet na tworzenie skryptów, zastępujących języki skryptowe takie jak \textsl{Bash}, \textsl{Perl}, \textsl{Python}

Język \textsl{Go}, podobnie jak \textsl{Java}, jest językiem statycznie typowanym. To znaczy, że typy zmiennych są znane w momencie kompilacji. Dzięki temu można uniknąć wielu błędów już w momencie kompilacji. Jednak język ma też wbudowane mechanizmy programowania dynamicznego. 

Dostępnie typy zmiennych są w większości podobne do tych, które występują w języku \textsl{C}. Występują tu również struktury i interfejsy. Jednak język pozwala na tworzenie metod (prywatnych i publicznych), które będą przypisane do struktury. Nie można powiedzieć, że język ten należy do grupy języków obiektowych ponieważ niemożliwe jest np. zaprogramowanie dziedziczenia. Kolejną różnicą w stosunku do języka \textsl{Java} jest brak typów generycznych, które tak bardzo pomogły w tworzonych systemach.
 
Składnia języka \textsl{Go} jest bardzo czytelna i łatwa do zrozumienia. Autorzy stworzyli zbiór standardów dotyczący m.in. wcięć czy wywoływania metod, które mają na celu uspójnienie pisanych programów. Kompilator języka ma wbudowany analizator, który w momencie odstępstw przerywa dalszą kompilację. Takie podejście pozwala, by programista mógł wdrożyć się w istniejący system skupiają się na jego działaniu, a nie dodatkowo na stylu składni jaki panuje w zespole

Język \textsl{Go} jest coraz częściej wykorzystywany np. w systemach  \textsl{Docker}, \textsl{Kubernetes}. Znalazł również zastosowanie w usługach sieciowych takich jak serwery DNS czy bazy danych.