\chapter{Przegląd literatury}

\section{Serwisy RESTful}
\subsection{Czym jest serwis RESTful}
\subsection{Mikroserwisy}

\section{Java}
\subsection{Historia i ewolucja języka Java}
Początki języka Java miały miejsce w 1991 roku. Wtedy to Patrick Naughton i James Gosling, inżynierowie firmy Sun,  wpadli na pomysł zaprojektowania języka, który będzie można uruchomić na urządzeniach takich jak dekodery telewizyjne\cite{java8}. Ponieważ urządzenia takie pochodziły od różnych producentów i każdy mógł stosować różne procesory, język nie mógł być oparty na jednej architekturze. W ten sposób zespół pracujący na rozwiązaniem wskrzesił model uruchamiania oprogramowania na maszynach wirtualnych, gdzie programy były kompilowane do kodu pośredniego.  
Początkowo język nosił nazwę Oak. Ponieważ taki język już istniał, postanowiono zmienić nazwę na Java.

Do roku 1994 roku zespół projektowy bez skutku próbował wykorzystać język w urządzeniach kablowych. Wtedy postanowili wykorzystać język do stworzenia przeglądarki internetowej, która będzie niezależna od architektury. W 1995 zaprezentowali oni efekt swojej pracy: przeglądarkę HotJava. Przez wbudowaną przenośność język został uznany, że jest wart rozwijania by w 1996 roku wydano wersję 1.0. Krótko po tym wyszła wersja 1.1 rozszerzająca możliwości m.in o możliwośc drukowania, dodano model zdarzeń dla programowania GUI oraz poprawiono refleksję.

Język miał wciąż spore braki i w 1998 roku wydano wersję 1.2. Wersja ta była ogromną zmianą w porównaniu do poprzedniczek, dlatego zmieniła swoją marketingową nazwę na "Java 2 Standard Edition Software Development Kit Version 1.2". Wraz z nią pojawiło się wiele nowych elementów. Do najważniejszych z nich należą biblioteka Collection oraz Swing. Biblioteka Swing pozwalała na tworzenie aplikacji z graficznymi interfejsami użytkownika(w tym apletów w przeglądarce), której elementy w łatwy sposób skalowały się do ekranów urządzeń na których była uruchamiana. W tym samym momencie swój początek miały dwie wersje języka. Micro Edition przeznaczona do urządzeń przenośnych np. telefonów komórkowych oraz Enterprise Edition do pisania aplikacji serwerowych. Kolejne wersje języka, 1.3 i 1.4 wydane odpowiednio w 2000 i 2002 roku, wprowadzały ulepszenia w samym języku oraz rozszerzały standardową bibliotekę. Z biegiem czasu popularność aplikacji uruchamianych po stronie klienta gasła, jednak rosła popularność pisania aplikacji po stronie uruchamianych na serwerach.

W 2004 pojawiła się nowa wersja języka oznaczona numerem 5.0. Wprowadziła ona typy sparametryzowane(ang. \textsl{generic types}) co rozszerzyło możliwości pisania oprogramowania, bardziej odpornego na błędy programisty. Doszły różwnież rzeczy zaczerpnięte z języka C\#: pętla \textsl{for each} do poruszania się po tabliach i kolekcjach, oraz \textsl{autoboxing} pozwalający konwertować typy proste na obiekty i odwrotnie. Dodane zmiany były zaimplementowane tak, by nie zmieniać nic w maszynie wirtualnej.

Wersja Java SE 6 nie wnosiła nowych funkcji. Rozszerzała ona jednak bibliotekę standardową, oraz poprawiono wydajność i bezpieczeństwo.

W wyniku problemów finansowych firmy Sun, koncern Oracle w 2010 wykupił firmę i w ten sposób stał się właścicielem praw do języka Java. Mocno opóźniona kolejna wersja wyszła w 2011. Wprowadzała ona sporo zmian. Do najważniejszych z nich należało dodanie frameworka \textsl{Fork/Join} upraszający programowanie równoległe, pozwalając na tworzenie procesów oraz poprawa wydajności na systemach z procesorami wielordzeniowymi. Kolejną sporą nowością była nowa biblioteka obsługi wejścia/wyjścia, nazwana NIO. Java 7 wnosiła też mechanizm \textsl{invokedynamic}. Mechanizm ten pozwalał na nowy sposób wywołania metod. Zmiana wymusiła zmiany w \textsl{bytecode} języka. Zmiana miała się stać fundamentem, do tego co przyniesie Java 8.

\subsection{Java 8}
Obecnie istniejącą wersją języka Java jest wersja numer 8. Została opublikowana w 2014 roku przynosząc długo oczekiwane zmiany.\\*
Najważniejszą z nich i najbardziej znaną były wyrażenia \textsl{Lambda}. Wyrażenia te pozwalają na czyste i zwięzłe wyrażenie pojedynczej metody. Wyrażenia te są wykorzystywane często przy operowaniu na kolekcjach między innymi do sortowania czy filtrowania kolekcji.\\*
Przykłady \ref{lst:javaanonymous} i \ref{lst:javalambda} przedstawiają sortowanie kolekcji z użyciem klas anonimowych oraz wyrażenia \textsl{Lambda}. 

\begin{lstlisting}[caption=Przykład sortowania kolekcji w języku Java przy użyciu klas anonimowych, label={lst:javaanonymous}]
Collections.sort(persons, new Comparator<Person>(){
  public int compare(Person p1, Person p2){
    return p1.firstName.compareTo(p2.firstName);
  }
});
\end{lstlisting}
\begin{lstlisting}[caption=Przykład sortowania kolekcji w języku Java przy użyciu wyrażeń Lambda,label={lst:javalambda}]
Collections.sort(persons, (p1, p2) -> p1.firstName.compareTo(p2.firstName));
\end{lstlisting}
Kod przy użyciu wyrażeń \textsl{Lambda} jest prostszy i krótszy.\\*
Wyrażenia te w \textsl{bytecode} używają mechanizmu \textsl{invokedynamic} wprowadzonego w Javie 7. Wykorzystanie tego mechanizmu pozwala na opóźnienie przetłumaczenia wyrażenia w \textsl{bytecode} do momentu ich wywołania. Wynikiem wyrażenia \textsl{Lambda} jest metoda statyczna, która jest tworzona w czasie wykonania \textsl{bytecode}.\\*
Kolejną sporą zmianą jest możliwość tworzenia metod domyślnych(ang. \textsl{default methods}) i statycznych(ang. \textsl{static methods}). Pozwoliło to m.in. na dostarczenie implementacji metody, bez potrzeby tworzenia klas abstrakcyjnych.

Dużą nowością w tej wersji były nowe klasy do obsługi dat i godzin. Pozwalają one na operacje takich jak dodawanie czy odejmowanie w sposób uporządkowany i bardziej naturalny do zrozumienia.\\*
Z mniej popularnych nowości jest możliwość wywoływania kodu \textsl{JavaScript} na wirtualnej maszynie Java(ang. \textsl{Java Virtual Machine}, w skrócie JVM).


\subsection{Biblioteka Spring}
\subsection{Kontenery aplikacji}
\subsubsection{Tomcat8}
\subsubsection{Jetty9}

\section{Go}
\subsection{Historia i ewolucja języka Go}
\subsection{Biblioteka mgo}
