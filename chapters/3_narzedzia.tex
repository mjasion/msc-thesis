\chapter{Narzędzia wykorzystane do przeprowadzenia testów}

\section{Docker}
Docker jest platforma dla programistów i administratorów systemóœ do tworzenia, dostarczania i uruchamiania aplikacji. Docker pozwala zbudować aplikację z zależnościami, która zachowywać się będzie tak samo na środowisku produkcyjnym jak i programistycznym. Dzieje się tak, ponieważ budując aplikację tworzymy obrazy, które po zbudowaniu przenosimy na docelowe środowisko.\\*
Docker opiera się na kontenerach linuxowych(LXC - Linux Containers). Kontenery w Linuxie są wirtualizacją na poziome systemu operacyjnego, która umożliwia na separacje aplikacji od systemu operacyjnego i fizycznej infrastruktury wykorzystywanej m.in. do połączeń sieciowych czy plików. Każdy z kontenerów może uruchomić swój proces, może mieć własnych użytkowników. Kontenerów w jednym systemie może być uruchomiona nieograniczona ilość. Konteneryzacja w przeciwieństwie do wirtualizacji  oferuje niewielki narzut na zasoby. Uruchomienie pojedynczego kontenera ogranicza się do wykonania kilku standardowych poleceń systemowych. 
Dla niniejszej pracy użyto Docker w wersji 1.9.0.

\section{MongoDB}
MongoDB(https://www.mongodb.org/)jest nierelacyjną bazą danych(ang. \textsl{NoSQL database}). Główną cechą tej bazy jest brak ściśle zdefiniowanej struktury. Dane w bazie przechowywane są w postaci dokumentów. Dokument jest strukturą złożoną z par klucz-wartość, przypominających obiekt JSON. Wartości w dokumencie mogą zawierać inne dokumenty, tablice czy tablice dokumentów:
\begin{lstlisting}[language=JavaScript,caption=Przykład dokumentu w formacie JSON]
{
    firstname: "Jan",
    lastname: "Kowalski",
    age: 40
}
\end{lstlisting}


\section{Apache JMeter}
Apache JMeter(http://jmeter.apache.org/) jest programem służącym do wykonywania testów aplikacji w celu zmierzenia jej wydajności.  Początkowo został stworzony do tworzenia testów serwisów internetowych. Jednak z czasem został on rozszerzony o dodatkowe funkcje. Apache JMeter można użyć do symulowania wysokiego obciążenia aplikacji na serwerze, sieci lub innych testowanych obiektach.
Obecnie JMeter można zastosować do testowania serwerów i protokołów:
\begin{itemize}
\item HTTP
\item HTTPS
\item FTP
\item SOAP oraz REST
\item relacyjne bazy danych - przy użyciu sternika JDBC
\item nierelacyjne bazy danych np MongoDB
\item usług pocztowych wykorzstujących protokoły: SMTP, POP3 oraz IMAP
\item TCP
\end{itemize}
Apache JMeter można rozszerzać o własne pluginy więc lista usług dostępnych do testowania jest nieograniczona. \\*
Apache JMeter jest wielowątkowym narzędziem, przez co można wykonywać ten sam test rówlnolegle, symulując w ten sposób wielu urzytkowników. \\
% TODO Jak tworzyć testy
Pierwsza stabilna wersja Apache JMeter została wydana 15 grudnia 1998r. Dla testów w pracy użyto wersji 2.13, która została wydana 14 marca 2015r.

\section{Digitalocean}
