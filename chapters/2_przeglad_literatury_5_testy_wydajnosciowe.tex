Ważnym procesem podczas tworzenia oprogramowanie jest jego testowanie. Celem testów jest zapewnienie jakości wytwarzanego oprogramowania zgodnego ze specyfikacją. Jednym z typów takich testów są testy wydajnościowe. Testy wydajnościowe pozwalają sprawdzić, jak oprogramowanie zachowa się w zależności od obciążenia. Między innymi można zbadać czy nie nastąpi przerwa w działaniu systemu, ile czasu będą zajmować poszczególne funkcje systemu, jak wykorzystane będą zasoby systemu oraz czy pozwala na skalowanie. 

Testy wydajnościowe mogą przyjmować różną formę. Pierwszą z nich są testy obciążeniowe. W takim tescie ustawia wybiera się ilu użytkowników jednocześnie ma korzystać z aplikacji. Na podstawie dancyh otrzymanych z takiego testu można zbadać, jak aplikacja zachowa się podczas określonego obciążenia Dzięki takim testom można otrzymać informacje o czasach odpowiedzi aplikacji.

Drugą formą są testy przeciążeniowe pozwalają na określenie, jak zachowa się system w warunkach skrajnego obciążenia. Przykładem takiego testu jest zbyt duża liczba użytkowników aplikacji, korzystających w tym samym czasie. Dzięki takim testom można między zaobserwować czy i w jakim momencie aplikacja się wyłączy.

Testy wydajnościowe mogą być prowadzone w różnym celu. W aplikacjach internetowych testy przeprowadza się by zbadać, jak dużo użytkowników jest w stanie z systemu korzystać jednocześnie. Takie testy można przeprowadzić zarówno z jednego komputera jak i z kilku. 

W aplikacjach badane również mogą być czasy odpowiedzi aplikacji w zależności od liczby równoległych żądań. Zazwyczaj testy takie są stosowane, gdy aplikacja w swojej specyfikacji ma określony maksymalny czas odpowiedzi przy danej liczbie żądań.

Testy wydajnościowe powinny być prowadzone na środowisku testowym, identycznym do produkcyjnego.