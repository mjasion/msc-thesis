\chapter{Wnioski}

Przedmiotem pracy było opracowanie i zaimplementowanie aplikacji, w dwóch różnych językach programowania (\textsl{Java} i \textsl{Go}), które posłużą do przeprowadzenia testów wydajnościowych. Stworzone aplikacje były aplikacjami \textsl{RESTful}, pozwalającymi na komunikacje z nimi przy pomocy protokołu \textsl{HTTP}. Aplikacje do działania wykorzystywały dodatkowo bazę danych \textsl{MongoDB}. Testy wydajnościowe zostały przeprowadzone na specjalnie przygotowanym środowisku.

Coś tu.....

Składnie obu języków od siebie się różnią. Język \textsl{Java} jest językiem obiektowym. \textsl{Go} natomiast pozwalają na stosowanie niewielu elementów obiektowości. Również obsługa błędów jest inna w obu językach. \textsl{Java} posiada wyjątki, natomiast w \textsl{Go} obsługa błędnych sytuacji sprowadza się to zwracania obiektu typu \textsl{Error}.
