\chapter{Wyniki testów}
Wyniki testów każdej aplikacji uruchamianej w poszczególnych przypadkach  testowych przedstawiono w formie diagramów: diagramu przedstawiającego liczbę żądań obsłużonych przez aplikację w ciągu sekundy i diagramu rozkładu czasów odpowiedzi aplikacji.

Wyniki testów podzielono na dwie grupy zależne  od początkowego stanu bazy danych.

\section{Testy z pustą bazą danych}

\subsection{Test wydajności walidacji API}
Wyniki testów wydajności walidujących istnienie klucza API przedstawiają diagramy zamieszczone na rysunkach \ref{fig:tomcat_clean_api_validation_rps} - \ref{fig:go_clean_api_validation_td}.

Z diagramów prezentujących liczbę żądań obsłużonych w ciągu sekundy przez poszczególne aplikacje (rys. \ref{fig:tomcat_clean_api_validation_rps}, \ref{fig:jetty_clean_api_validation_rps}, \ref{fig:go_clean_api_validation_rps}) wynika, że przy 100 klientach największą liczbę żądań obsłużyła aplikacja napisana w \textsl{Go} - od 9.0 do 11.0 tysięcy obsłużonych żądań. Wydajność serwera \textsl{Jetty} oscylowała w przedziale od 5.0 do 10.0 tysięcy obsłużonych żądań, przy znacznych wahaniach przepustowości. Przepustowość serwera Tomcat oscylowała w przedziale od 6.0 do 8.0 tysięcy obsłużonych żądań. Przy 250 klientach wydajność aplikacji w Go oscylowała w przedziale od 8.0 do 10.0 tysięcy obsłużonych żądań, a serwery Jetty i Tomcat obsłużyły od 4.0 do 8.0 tysięcy żądań przy dużych wahaniach przepustowości.      

Z diagramów rozkładu czasów odpowiedzi aplikacji (rys. \ref{fig:tomcat_clean_api_validation_td}, \ref{fig:jetty_clean_api_validation_td}, \ref{fig:go_clean_api_validation_td}) wynika, że przy 100 klientach średnie czasy odpowiedzi wynosiły: dla serwera Tomcat 8 13.70 milisekund, dla serwera Jetty 11.42 milisekund, a dla aplikacji w Go 9,44 milisekundy. Najdłużej trwające żądania trwały około 35 milisekund w przypadku serwerów Tomcat i Jetty oraz 18 milisekund przy aplikacji w Go. Przy 250 klientach średnie czasy odpowiedzi aplikacji uruchamianych na serwerach Tomcat i Jetty były prawie takie same i wynosiły odpowiednio 36.32 i 35.52 milisekund, a w aplikacji w Go 22t.47 milisekundy. Najdłużej trwające żądania trwały ponad 90 milisekund w przypadku serwera Tomcat, poniżej 88 milisekund w przypadku serwera Jetty i około 55 milisekundy przy aplikacji w Go. Dodatkowo, przy 250 klientach, dla serwerów Tomcat i Jetty rozkłady czasów odpowiedzi były spłaszczone. 

% \pgfplotsset{grid style={dashed}}
\begin{figure}[!ht]
\pgfplotstableread[col sep = comma]{csv_queries/requests_per_sec/tomcat_clean_api_validation.csv}\csvdata
\begin{tikzpicture}
  \begin{axis}[xmin = 0, xmax=900, ymin = 0, scaled y ticks = base 10:-3, xlabel = {Czas [s]}, ylabel = Liczba żądań, legend pos=south east, ymajorgrids,width=13cm, height=6cm] %TODO miary?
    \addplot[color=blue,mark=none] table[x index=0, y index=1]{\csvdata};
    \addplot[color=green,mark=none] table[x index=0, y index=2]{\csvdata};
    \legend{100,250}
  \end{axis}
\end{tikzpicture}
\caption{Tomcat 8 - liczba żądań obsłużonych przez aplikację w ciągu sekundy podczas testu walidacji istnienia klucza API}
\label{fig:tomcat_clean_api_validation_rps}
\end{figure}

\begin{figure}[!ht]
\pgfplotstableread[col sep = comma]{csv_queries/response_time_distribution/tomcat_clean_api_validation_100.csv}\csva
\pgfplotstableread[col sep = comma]{csv_queries/response_time_distribution/tomcat_clean_api_validation_250.csv}\csvb
\pgfplotsset{
    /pgfplots/ybar legend/.style={
    /pgfplots/legend image code/.code={\draw[##1,/tikz/.cd,yshift=-0.25em](0cm,0cm) rectangle(1pt,0.7em);},
   }
}
\begin{tikzpicture}
  \begin{axis}[ybar, bar width=0.5, xmin = 0, ymin = 0, scaled y ticks = base 10:-5, xlabel = {Czas odpowiedzi [ms]}, ylabel = {Liczba żądań}, ymajorgrids,width=13cm, height=6cm] %TODO miary?
    \addplot[color=blue, mark=none, fill=blue] table[x index=0, y index=1]{\csva};
    \addplot[color=green, mark=none, fill=green] table[x index=0, y index=1]{\csvb};
    \legend{100,250}
  \end{axis}
\end{tikzpicture}
\caption{Tomcat 8 - rozkład czasów odpowiedzi aplikacji (95\% odpowiedzi) podczas testu walidacji istnienia klucza API}
\label{fig:tomcat_clean_api_validation_td}
\end{figure}

\pgfplotsset{grid style={dashed}}
\begin{figure}[!ht]
\pgfplotstableread[col sep = comma]{csv_queries/requests_per_sec/jetty_clean_api_validation.csv}\csvdata
\begin{tikzpicture}
  \begin{axis}[xmin = 0, xmax=900, ymin = 0, scaled y ticks = base 10:-3, xlabel = {Czas [s]}, ylabel = Liczba żądań, legend pos=south east, ymajorgrids,width=13cm, height=6cm] %TODO miary?
    \addplot[color=blue,mark=none] table[x index=0, y index=1]{\csvdata};
    \addplot[color=green,mark=none] table[x index=0, y index=2]{\csvdata};
    \legend{100,250}
  \end{axis}
\end{tikzpicture}
\caption{Jetty 9 - liczba żądań obsłużonych przez aplikację w ciągu sekundy podczas testu walidacji istnienia klucza API}
\label{fig:jetty_clean_api_validation_rps}
\end{figure}

\begin{figure}[!ht]
\pgfplotstableread[col sep = comma]{csv_queries/response_time_distribution/jetty_clean_api_validation_100.csv}\csva
\pgfplotstableread[col sep = comma]{csv_queries/response_time_distribution/jetty_clean_api_validation_250.csv}\csvb
\pgfplotsset{
    /pgfplots/ybar legend/.style={
    /pgfplots/legend image code/.code={\draw[##1,/tikz/.cd,yshift=-0.25em](0cm,0cm) rectangle(1pt,0.7em);},
   }
}
\begin{tikzpicture}
  \begin{axis}[ybar, bar width=0.5, xmin = 0, ymin = 0, scaled y ticks = base 10:-5, xlabel = {Czas odpowiedzi [ms]}, ylabel = {Liczba żądań}, ymajorgrids,width=13cm, height=6cm] %TODO miary?
    \addplot[color=blue, mark=none, fill=blue] table[x index=0, y index=1]{\csva};
    \addplot[color=green, mark=none, fill=green] table[x index=0, y index=1]{\csvb};
    \legend{100,250}
  \end{axis}
\end{tikzpicture}
\caption{Jetty 9 - rozkład czasów odpowiedzi aplikacji (95\% odpowiedzi) podczas testu walidacji istnienia klucza API}
\label{fig:jetty_clean_api_validation_td}
\end{figure}

\pgfplotsset{grid style={dashed}}
\begin{figure}[!ht]
\pgfplotstableread[col sep = comma]{csv_queries/requests_per_sec/go_clean_api_validation.csv}\csvdata
\begin{tikzpicture}
  \begin{axis}[xmin = 0, xmax=900, ymin = 0, scaled y ticks = base 10:-3, xlabel = {Czas [s]}, ylabel = Liczba żądań, legend pos=south east, ymajorgrids,width=13cm, height=6cm] %TODO miary?
    \addplot[color=blue,mark=none] table[x index=0, y index=1]{\csvdata};
    \addplot[color=green,mark=none] table[x index=0, y index=2]{\csvdata};
    \legend{100,250}
  \end{axis}
\end{tikzpicture}
\caption{Go - liczba żądań obsłużonych przez aplikację w ciągu sekundy podczas testu walidacji istnienia klucza API}
\label{fig:go_clean_api_validation_rps}
\end{figure}

\begin{figure}[!ht]
\pgfplotstableread[col sep = comma]{csv_queries/response_time_distribution/go_clean_api_validation_100.csv}\csva
\pgfplotstableread[col sep = comma]{csv_queries/response_time_distribution/go_clean_api_validation_250.csv}\csvb
\pgfplotsset{
    /pgfplots/ybar legend/.style={
    /pgfplots/legend image code/.code={\draw[##1,/tikz/.cd,yshift=-0.25em](0cm,0cm) rectangle(1pt,0.7em);},
   }
}
\begin{tikzpicture}
  \begin{axis}[ybar, bar width=0.5, xmin = 0, ymin = 0, scaled y ticks = base 10:-5, xlabel = {Czas odpowiedzi [ms]}, ylabel = {Liczba żądań}, ymajorgrids,width=13cm, height=6cm] %TODO miary?
    \addplot[color=blue, mark=none, fill=blue] table[x index=0, y index=1]{\csva};
    \addplot[color=green, mark=none, fill=green] table[x index=0, y index=1]{\csvb};
    \legend{100,250}
  \end{axis}
\end{tikzpicture}
\caption{Go - rozkład czasów odpowiedzi aplikacji (95\% odpowiedzi) podczas testu walidacji istnienia klucza API}
\label{fig:go_clean_api_validation_td}
\end{figure}


\clearpage

\subsection{Test wydajności walidacji istnienia obiektów Cache}
Wyniki testów wydajności walidacji istnienia obiektów Cache przedstawiają diagramy na rysunkach \ref{fig:tomcat_clean_key_validation_rps} - \ref{fig:go_clean_key_validation_td}.              

Z diagramów przedstawiających rozkład ilości żądań obsłużonych przez poszczególne aplikacje  (rys. \ref{fig:tomcat_clean_key_validation_rps}, \ref{fig:jetty_clean_key_validation_rps}, \ref{fig:go_clean_key_validation_rps}) wynika, że przy 100 klientach największą liczbę żądań (około 7.0 tysięcy) obsłużyła aplikacja napisana w \textsl{Go}. Wydajność serwera \textsl{Jetty} oscylowała w przedziale od 3.5 do 4.5 tysiąca obsłużonych żądań, a serwera Tomcat w przedziale od 2.0 do 3.5 tysiąca obsłużonych żądań. Przy 250 klientach wydajność aplikacji w Go oscylowała w przedziale od 7.0 do 8.0 tysięcy obsłużonych żądań, serwer Jetty obsłużył około 3.0 tysięcy żądań, a serwer Tomcat od 2.0 do 3.5 tysiąca żądań.

Z diagramów rozkładu czasów odpowiedzi (rys. \ref{fig:tomcat_clean_key_validation_td}, \ref{fig:jetty_clean_key_validation_td}, \ref{fig:go_clean_key_validation_td}) wynika, że przy 100 klientach średnie czasy odpowiedzi wynosiły odpowiednio: dla serwera Tomcat 30.32 milisekund, dla serwera Jetty 23.32 milisekund, a dla aplikacji w Go 13.79 milisekund. Najdłużej trwające żądania trwały około 120 milisekund w przypadku serwera Tomcat, 45 milisekund przy Jetty i tylko 25 milisekund w aplikacji w Go. Przy 250 klientach średnie czasy odpowiedzi aplikacji uruchamianych na serwerach Tomcat i Jetty wynosiły odpowiednio 69.97 i 76.92 milisekundy, a w aplikacji w Go tylko 30.75 milisekundy. Najdłużej trwające żądania trwały 250 milisekund w przypadku serwera Tomcat, okło 120 milisekund w przypadku serwera Jetty i tylko 65 milisekundy przy aplikacji w Go. Czasy odpowiedzi serwera Tomcat w sporej części o były dużo dłuższe niż pokazuje to średnia.

\input{chapters/5_testy_wydajnosciowe_diagram_2_clean_key_validation.tex}
\clearpage

\subsection{Test wydajności operacji CRUD}
Wyniki testów wydajności operacji CRUD przedstawiają diagramy na rysunkach 6.13 – 6.18.

Z diagramów przedstawiających rozkład ilości żądań obsłużonych przez poszczególne aplikacje w ciągu sekundy (rys. \ref{fig:tomcat_clean_key_validation_rps}, \ref{fig:jetty_clean_key_validation_rps}, \ref{fig:go_clean_key_validation_rps}) wynika, że przy przy 100 klientach największą liczbę żądań obsłużyła aplikacja napisana w \textsl{Go} - od 4.0 do 5.0 tysięcy. Serwer \textsl{Jetty} obsługiwał ponad 2.2 tysiące żądań, a przepustowość serwera Tomcat oscylowała w przedziale od 2.0 do 2.5 tysiąca obsłużonych żądań. Przy 250 klientach aplikacja w Go obsłużyła od 4.5 do 6.0 tysięcy żądań, serwer Jetty około 2.0 tysięcy żądań, podobnie jak serwer Tomcat.

Z diagramów rozkładu czasów odpowiedzi (rys. \ref{fig:tomcat_clean_key_validation_td}, \ref{fig:jetty_clean_key_validation_td}, \ref{fig:go_clean_key_validation_td}) wynika, że przy 100 klientach średnie czasy odpowiedzi wynosiły odpowiednio: dla serwera Tomcat 42.15  milisekund, dla serwera Jetty 42.41  milisekund i 20.90 milisekund dla aplikacji w Go.  Najdłużej trwające żądania trwały poniżej 80 milisekund w przypadku serwerów Tomcat i Jetty - jednak \textsl{Jetty} miał większą liczbę obsłużonych żądań. Dla aplikacji w \textsl{Go} najdłużej trwające żądania trwały tylko 35 milisekund. Przy 250 klientach średnie czas odpowiedzi aplikacji uruchamianych na serwerach Tomcat i Jetty  wynosiły  odpowiednio 107.56 i 92.32 milisekund, a w aplikacji w Go tylko 35.68 milisekundy. Najdłużej trwające żądania trwały 200 milisekund w przypadku serwera Tomcat, 180 milisekund w przypadku serwera Jetty i 90 milisekund przy aplikacji w Go. 

\input{chapters/5_testy_wydajnosciowe_diagram_3_clean_crud.tex}
\clearpage

\subsection{Test wydajności walidacji API, obiektów Cache oraz operacji CRUD równolegle}
Diagramy zawierające wyniki wydajności walidacji API, obiektów Cache oraz operacji CRUD równolegle zamieszczono na rys. 6.19 – 6.24.                                                                                            
Z rozkładów ilości żądań obsłużonych  przez poszczególne aplikacje (rys. 6.19, 6.21, 6.23) wynika, że przy 100 klientach największą liczbę żądań  obsłużyła aplikacja napisana w \textsl{Go}- od 7.0 do 8.0 tysięcy, a serwery \textsl{Jetty 9}i Tomcat 8 obsłużyły od 3.0 do 4.0  tysiący żądań. Przy 250 klientach aplikacja w Go obsłużyła również od 7.0 do 8.0 tysięcy żądń, serwer Jetty 9 3.0 tysiące, a wydajność serwera Tomcat 8 oscylowała na poziomie 2.5 tysiąca obsłużonych żądań. 
 
Z diagramów rozkładu czasów odpowiedzi (rys. \ref{fig:tomcat_clean_key_validation_td}, \ref{fig:jetty_clean_key_validation_td}, \ref{fig:go_clean_key_validation_td}) wynika, że przy 100 klientach średnie czasy odpowiedzi wynosiły odpowiednio: dla serwera Tomcat 8 [] milisekund, dla serwera Jetty 9 [] milisekund i [] milisekund dla aplikacji w Go.  Najdłużej trwające żądania trwały około 110 milisekund w przypadku serwera Tomcat 8, 150 milisekund przy Jetty 9 i 35 milisekund  w aplikacji w Go. Przy 250 klientach średnie czas odpowiedzi aplikacji uruchamianych na serwerach Tomcat i Jetty 9  wynosiły  odpowiednio [[]i [] milisekundy, a w aplikacji w Go tylko [] milisekundy. Najdłużej trwające żądania trwały 180 milisekund w przypadku serwera Tomcat 8, 150 milisekund w przypadku serwera Jetty 9 i 85 milisekund przy aplikacji w Go. 

% \input{chapters/5_testy_wydajnosciowe_diagram_4_clean_all.tex}
\clearpage

\newpage
\section{Testy z bazą wypełnioną danymi początkowymi}
\subsection{Testy wydajności walidacji API}
Diagramy zawierające wyniki wydajności walidacji API zamieszczono na rys.
Z rozkładów ilości żądań obsłużonych przez poszczególne aplikacje (rys. 6.25, 6.27, 6.29) wynika, że przy 100 klientach aplikacja napisana w \textsl{Go} obsłużyła około 9.0 tysięcy żądań. Wydajność serwera  \textsl{Jetty 9} oscylowała w przedziale od 8.0 do 10.0 tysięcy obsłużonych żądań. Serwer Tomcat 8  obsłużył od 4.0 do 7.0 tysiący żądań z dużymi wahaniami przepustowości. Przy 250 klientach aplikacja w Go obsłużyła około od 8.0 do 10.0 tysięcy żądań, serwer Jetty 3 od 4.0 do 10.0 tysięcy żądań przy bardzo dużych wahaniach przepustowości a wydajność serwera Tomcat 8 oscylowała w przedziale od 3.0 do 4.0 tysięcy  obsłużonych żądań. 
 
Z diagramów rozkładu czasów odpowiedzi (rys. \ref{fig:tomcat_clean_key_validation_td}, \ref{fig:jetty_clean_key_validation_td}, \ref{fig:go_clean_key_validation_td}) wynika, że przy 100 klientach średnie czasy odpowiedzi wynosiły odpowiednio: dla serwera Tomcat 8 [] milisekund, dla serwera Jetty 9 [] milisekund i [] milisekund dla aplikacji w Go.  Najdłużej trwające żądania trwały około 45 milisekund przy serwerze Tomcat 8, 30 milisekund przy Jetty 9 i 18 milisekund  w aplikacji w Go. Przy 250 klientach średnie czasy odpowiedzi aplikacji uruchamianych na serwerach Tomcat 8 i Jetty 9  wynosiły odpowiednio [[]i [] milisekundy, a w aplikacji w Go tylko [] milisekundy. Najdłużej trwające żądania trwały 105 milisekund w przypadku serwera Tomcat 8, 80 milisekund w przypadku serwera Jetty 9 przy spłaszczonym rozkłazie i 55 milisekund przy aplikacji w Go. 

% \input{chapters/5_testy_wydajnosciowe_diagram_5_full_api_validation.tex}
\clearpage

\subsection{Test wydajności walidacji istnienia obiektów Cache}
Wyniki testów wydajności walidacji istnienia obiektów Cache przedstawiają diagramy na rysunkach \ref{fig:tomcat_clean_key_validation_rps} - \ref{fig:go_clean_key_validation_td}.              

Z diagramów przedstawiających rozkład ilości żądań obsłużonych przez poszczególne aplikacje  (rys. \ref{fig:tomcat_clean_key_validation_rps}, \ref{fig:jetty_clean_key_validation_rps}, \ref{fig:go_clean_key_validation_rps}) wynika, że przy 100 klientach przepustowość aplikacjia w \textsl{Go} kształtowała się w przedziale od 0.0 do 6.0 tysięcy obsłużonych żądań. Wydajność serwera \textsl{Jetty 9} oscylowała w przedziale od 4.0 do 5.0 tysięcy obsłużonych żądań, a  serwera Tomcat 8  od 2.0 do 3.0 tysięcy obsłużonych żądń.  Przy 250 klientach aplikacja w Go obsłużyła od 1.0 do 5.0 tysięcy żądń, serwer Jetty 9 około 3.0 tysięcy żądań, a serwer Tomcat 8 od 2.0 do 3.0 tysiąca żądań.

Z diagramów rozkładu czasów odpowiedzi (rys. \ref{fig:tomcat_clean_key_validation_td}, \ref{fig:jetty_clean_key_validation_td}, \ref{fig:go_clean_key_validation_td}) wynika, że przy 100 klientach średnie czasy odpowiedzi wynosiły odpowiednio: dla serwera Tomcat 8 [] milisekund, dla serwera Jetty 9 [] milisekund i [] milisekund dla aplikacji w Go.  Najdłużej trwające żądania trwały 150 milisekund przy serwerze Tomcat 8, 40 milisekund, z symetrycznym rozkładem przy serwerze  Jetty 9 i 40 milisekund  w aplikacji w Go. Przy 250 klientach średnie czasy odpowiedzi aplikacji uruchamianych na serwerach Tomcat 8 i Jetty 9  wynosiły odpowiednio [[]i [] milisekundy, a w aplikacji w Go tylko [] milisekundy. Najdłużej trwające żądania trwały 250 milisekund w przypadku serwera Tomcat 8, 120 milisekund w przypadku serwera Jetty 9 i 95 milisekund przy aplikacji w Go. 

\input{chapters/5_testy_wydajnosciowe_diagram_6_full_key_validation.tex}
\clearpage

\subsection{Test wydajności operacji CRUD}

Wyniki testów wydajności walidacji istnienia operacji CRUD przedstawiają diagramy na rysunkach \ref{fig:tomcat_clean_key_validation_rps} - \ref{fig:go_clean_key_validation_td}.              

Z diagramów przedstawiających rozkład ilości żądań obsłużonych przez poszczególne aplikacje  (rys. \ref{fig:tomcat_clean_key_validation_rps}, \ref{fig:jetty_clean_key_validation_rps}, \ref{fig:go_clean_key_validation_rps}) wynika, że przy 100 i 250 klientach aplikacje zachowały się  porównywalńie. Przepustowość aplikacji w \textsl{Go} oscylowała na poziomie 3.0 tysięcy obsłużonych żądań. Serwera \textsl{Jetty 9} obsłużył około 2.0 tysięcy  żądań, a serwer Tomcat 8 około 1.5 tysięca żądań.


Z diagramów rozkładu czasów odpowiedzi (rys. \ref{fig:tomcat_clean_key_validation_td}, \ref{fig:jetty_clean_key_validation_td}, \ref{fig:go_clean_key_validation_td}) wynika, że przy 100 klientach średnie czasy odpowiedzi wynosiły odpowiednio: dla serwera Tomcat 8 [] milisekund, dla serwera Jetty 9 [] milisekund i [] milisekund dla aplikacji w Go.  Najdłużej trwające żądania trwały 150 milisekund przy serwerze Tomcat 8, 110 milisekund przy serwerze  Jetty 9 i 70 milisekund  w aplikacji w Go. Przy 250 klientach średnie czasy odpowiedzi aplikacji uruchamianych na serwerach Tomcat 8 i Jetty 9  wynosiły odpowiednio [[]i [] milisekundy, a w aplikacji w Go tylko [] milisekundy. Najdłużej trwające żądania trwały 300 milisekund w przypadku serwera Tomcat 8, 240 milisekund w przypadku serwera Jetty 9 i 170. milisekund przy aplikacji w Go. 

% \input{chapters/5_testy_wydajnosciowe_diagram_7_full_crud.tex}
\clearpage

\subsection{Test wydajności walidacji API, obiektów Cache oraz operacji CRUD równolegle }

Wyniki testów wydajności walidacji API, obiektów Cache oraz operacji CRUD równolegle przedstawiają diagramy na rysunkach \ref{fig:tomcat_clean_key_validation_rps} - \ref{fig:go_clean_key_validation_td}.              

Z diagramów przedstawiających rozkład ilości żądań obsłużonych przez poszczególne aplikacje  (rys. \ref{fig:tomcat_clean_key_validation_rps}, \ref{fig:jetty_clean_key_validation_rps}, \ref{fig:go_clean_key_validation_rps}) wynika, że przy 100 klientach przepustowość aplikacjia w \textsl{Go} kształtowała się w przedziale od 4.0 do 6.0 tysięcy obsłużonych żądań. Wydajność serwera \textsl{Jetty 9} oscylowała w przedziale od 3.0 do 3.5 tysięca obsłużonych żądań, a  serwera Tomcat 8 obsłużył od 2.2 do 3.0 tysięcy żądań. Przy 250 klientach aplikacja w Go obsłużyła 4.0 do 6.0 tysięcy żądń. Przepustowość serwera Jetty 9 oscylowała w przedziale od 2.0 do 3.0 tysięcy obsłużonych żądań, a serwera Tomcat 8 w przedziale od 2.0 do 2.5 tysiąca obsłużonych żądań.


Z diagramów rozkładu czasów odpowiedzi (rys. \ref{fig:tomcat_clean_key_validation_td}, \ref{fig:jetty_clean_key_validation_td}, \ref{fig:go_clean_key_validation_td}) wynika, że przy 100 klientach średnie czasy odpowiedzi wynosiły odpowiednio: dla serwera Tomcat 8 [] milisekund, dla serwera Jetty 9 [] milisekund i [] milisekund dla aplikacji w Go. Najdłużej trwające żądania trwały 100 milisekund przy serwerze Tomcat 8, 80 milisekund przy serwerze  Jetty 9 i 50 milisekund  w aplikacji w Go. Przy 250 klientach średnie czasy odpowiedzi aplikacji uruchamianych na serwerach Tomcat 8 i Jetty 9  wynosiły odpowiednio [[]i [] milisekundy, a w aplikacji w Go tylko [] milisekundy. Najdłużej trwające żądania trwały 200 milisekund w przypadku serwera Tomcat 8 i Jetty 9 i 13 milisekund przy aplikacji w Go. 

% \pgfplotsset{grid style={dashed}}
\begin{figure}[!ht]
\pgfplotstableread[col sep = comma]{csv_queries/requests_per_sec/tomcat_full_all.csv}\csvdata
\begin{tikzpicture}
  \begin{axis}[xmin = 0, xmax=900, ymin = 0, scaled y ticks = base 10:-3, xlabel = {Czas [s]}, ylabel = Liczba żądań, legend pos=south east, ymajorgrids] %TODO miary?
    \addplot[color=blue,mark=none] table[x index=0, y index=1]{\csvdata};
    \addplot[color=green,mark=none] table[x index=0, y index=2]{\csvdata};
    \legend{100,250}
  \end{axis}
\end{tikzpicture}
\caption{Tomcat 8 - liczba żądań obsłużonych przez aplikację w ciągu sekundy podczas testu: walidacji istnienia klucza API, walidacji istnienia, operacji CRUD równolegle}
\label{fig:tomcat_full_all_rps}
\end{figure}

\begin{figure}[!ht]
\pgfplotstableread[col sep = comma]{csv_queries/response_time_distribution/tomcat_full_all_100.csv}\csva
\pgfplotstableread[col sep = comma]{csv_queries/response_time_distribution/tomcat_full_all_250.csv}\csvb
\pgfplotsset{
    /pgfplots/ybar legend/.style={
    /pgfplots/legend image code/.code={\draw[##1,/tikz/.cd,yshift=-0.25em](0cm,0cm) rectangle(1pt,0.7em);},
   }
}
\begin{tikzpicture}
  \begin{axis}[ybar, bar width=0.5, xmin = 0, ymin = 0, scaled y ticks = base 10:-5, xlabel = {Czas odpowiedzi [ms]}, ylabel = {Liczba żądań}, ymajorgrids] %TODO miary?
    \addplot[color=blue, mark=none, fill=blue] table[x index=0, y index=1]{\csva};
    \addplot[color=green, mark=none, fill=green] table[x index=0, y index=1]{\csvb};
    \legend{100,250}
  \end{axis}
\end{tikzpicture}
\caption{Tomcat 8 - rozkład czasów odpowiedzi aplikacji (95\% odpowiedzi) podczas testu: walidacji istnienia klucza API, walidacji istnienia, operacji CRUD równolegle}
\label{fig:tomcat_full_all_td}
\end{figure}

\pgfplotsset{grid style={dashed}}
\begin{figure}[!ht]
\pgfplotstableread[col sep = comma]{csv_queries/requests_per_sec/jetty_full_all.csv}\csvdata
\begin{tikzpicture}
  \begin{axis}[xmin = 0, xmax=900, ymin = 0, scaled y ticks = base 10:-3, xlabel = {Czas [s]}, ylabel = Liczba żądań, legend pos=south east, ymajorgrids] %TODO miary?
    \addplot[color=blue,mark=none] table[x index=0, y index=1]{\csvdata};
    \addplot[color=green,mark=none] table[x index=0, y index=2]{\csvdata};
    \legend{100,250}
  \end{axis}
\end{tikzpicture}
\caption{Jetty 9 - liczba żądań obsłużonych przez aplikację w ciągu sekundy podczas testu: walidacji istnienia klucza API, walidacji istnienia, operacji CRUD równolegle}
\label{fig:jetty_full_all_rps}
\end{figure}

\begin{figure}[!ht]
\pgfplotstableread[col sep = comma]{csv_queries/response_time_distribution/jetty_full_all_100.csv}\csva
\pgfplotstableread[col sep = comma]{csv_queries/response_time_distribution/jetty_full_all_250.csv}\csvb
\pgfplotsset{
    /pgfplots/ybar legend/.style={
    /pgfplots/legend image code/.code={\draw[##1,/tikz/.cd,yshift=-0.25em](0cm,0cm) rectangle(1pt,0.7em);},
   }
}
\begin{tikzpicture}
  \begin{axis}[ybar, bar width=0.5, xmin = 0, ymin = 0, scaled y ticks = base 10:-5, xlabel = {Czas odpowiedzi [ms]}, ylabel = {Liczba żądań}, ymajorgrids] %TODO miary?
    \addplot[color=blue, mark=none, fill=blue] table[x index=0, y index=1]{\csva};
    \addplot[color=green, mark=none, fill=green] table[x index=0, y index=1]{\csvb};
    \legend{100,250}
  \end{axis}
\end{tikzpicture}
\caption{Jetty 9 - rozkład czasów odpowiedzi aplikacji (95\% odpowiedzi) podczas testu: walidacji istnienia klucza API, walidacji istnienia, operacji CRUD równolegle}
\label{fig:jetty_full_all_td}
\end{figure}

\pgfplotsset{grid style={dashed}}
\begin{figure}[!ht]
\pgfplotstableread[col sep = comma]{csv_queries/requests_per_sec/go_full_all.csv}\csvdata
\begin{tikzpicture}
  \begin{axis}[xmin = 0, xmax=900, ymin = 0, scaled y ticks = base 10:-3, xlabel = {Czas [s]}, ylabel = Liczba żądań, legend pos=south east, ymajorgrids] %TODO miary?
    \addplot[color=blue,mark=none] table[x index=0, y index=1]{\csvdata};
    \addplot[color=green,mark=none] table[x index=0, y index=2]{\csvdata};
    \legend{100,250}
  \end{axis}
\end{tikzpicture}
\caption{Go - liczba żądań obsłużonych przez aplikację w ciągu sekundy podczas testu: walidacji istnienia klucza API, walidacji istnienia, operacji CRUD równolegle}
\label{fig:go_full_all_rps}
\end{figure}

\begin{figure}[!ht]
\pgfplotstableread[col sep = comma]{csv_queries/response_time_distribution/go_full_all_100.csv}\csva
\pgfplotstableread[col sep = comma]{csv_queries/response_time_distribution/go_full_all_250.csv}\csvb
\pgfplotsset{
    /pgfplots/ybar legend/.style={
    /pgfplots/legend image code/.code={\draw[##1,/tikz/.cd,yshift=-0.25em](0cm,0cm) rectangle(1pt,0.7em);},
   }
}
\begin{tikzpicture}
  \begin{axis}[ybar, bar width=0.5, xmin = 0, ymin = 0, scaled y ticks = base 10:-5, xlabel = {Czas odpowiedzi [ms]}, ylabel = {Liczba żądań}, ymajorgrids] %TODO miary?
    \addplot[color=blue, mark=none, fill=blue] table[x index=0, y index=1]{\csva};
    \addplot[color=green, mark=none, fill=green] table[x index=0, y index=1]{\csvb};
    \legend{100,250}
  \end{axis}
\end{tikzpicture}
\caption{Go - rozkład czasów odpowiedzi aplikacji (95\% odpowiedzi) podczas testu: walidacji istnienia klucza API, walidacji istnienia, operacji CRUD równolegle}
\label{fig:go_full_all_td}
\end{figure}


\clearpage

\section{Obciążenie maszyn wirtualnych}

\newpage
\section{Podsumowanie wyników}
