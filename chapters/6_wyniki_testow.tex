\chapter{Wyniki testów}
Wyniki testów każdej aplikacji uruchamianej w poszczególnych przypadkach  testowych przedstawiono w formie diagramów: diagramu przedstawiającego liczbę żądań obsłużonych przez aplikację w ciągu sekundy i diagramu rozkładu czasów odpowiedzi aplikacji.

Wyniki testów podzielono na dwie grupy, zależne od początkowego stanu bazy danych.

\section{Testy z pustą bazą danych}

\subsection{Test wydajności walidacji API}
Wyniki testów wydajności walidujących istnienie klucza API przedstawiają diagramy zamieszczone na rysunkach \ref{fig:tomcat_clean_api_validation_rps} - \ref{fig:go_clean_api_validation_td}. 
Z diagramów prezentujących liczbę żądań obsłużonych w ciągu sekundy przez poszczególne aplikacje (rys. \ref{fig:tomcat_clean_api_validation_rps}, \ref{fig:jetty_clean_api_validation_rps}, \ref{fig:go_clean_api_validation_rps}) wynika, że największą wydajność miała aplikacja napisana w \textsl{Go} – przy 100 klientach była w stanie obsłużyć około 10 tysięcy żądań,  a przy 250 klientach jej wydajność oscylowała w granicach 9 tysięcy żądań. Wydajność serwera \textsl{Jetty 9} z aplikacją Java oscylowała w przedziale 7 do 10 tysięcy żądań przy 100 klientach i 5 do 10 tysięcy żądań przy 250 klientach. Najmniejszą przepustowością wykazał się serwer Tomcat 8, również z aplikacją Java – jego przepustowość osiągnęła wartości między 6 a 8 tysięcy żądań przy 100 klientach i 4 do 8 tysięcy żądań przy 250 klientach.   
Z diagramów rozkładu czasów odpowiedzi aplikacji (rys. \ref{fig:tomcat_clean_api_validation_td}, \ref{fig:jetty_clean_api_validation_td}, \ref{fig:go_clean_api_validation_td}) wynika, że podczas próby 100 klientów najlepsze wyniki osiągnęła aplikacja  w Go - średni czasy odpowiedzi wyniósł 8.31 milisekund. Dla aplikacji Java uruchamianej na serwerze \textsl{Jetty 9} największa ilość żądań trwała poniżej 10 milisekund, a dla aplikacji uruchamianej na serwerze \textsl{Tomcat 8} średni czas odpowiedzi wyniósł 11.82 milisekund. W teście z 250 klientami, również najmniejszy średni czas odpowiedzi osiągnęła aplikacja  w \textsl{Go} (23,80 milisekundy).  Serwery \textsl{Jetty 9} i \textsl{Tomcat 8} miały porównywalne średnie czasy odpowiedzi (33,66 i 33,73 milisekund), ale w obu przypadkach rozkład czasów bardzo się spłaszczył. 

\pgfplotsset{grid style={dashed}}
\begin{figure}[!ht]
\pgfplotstableread[col sep = comma]{csv_queries/requests_per_sec/tomcat_clean_api_validation.csv}\csvdata
\begin{tikzpicture}
  \begin{axis}[xmin = 0, xmax=900, ymin = 0, scaled y ticks = base 10:-3, xlabel = {Czas [s]}, ylabel = Liczba żądań, legend pos=south east, ymajorgrids,width=13cm, height=6cm] %TODO miary?
    \addplot[color=blue,mark=none] table[x index=0, y index=1]{\csvdata};
    \addplot[color=green,mark=none] table[x index=0, y index=2]{\csvdata};
    \legend{100,250}
  \end{axis}
\end{tikzpicture}
\caption{Tomcat 8 - liczba żądań obsłużonych przez aplikację w ciągu sekundy podczas testu walidacji istnienia klucza API}
\label{fig:tomcat_clean_api_validation_rps}
\end{figure}

\begin{figure}[!ht]
\pgfplotstableread[col sep = comma]{csv_queries/response_time_distribution/tomcat_clean_api_validation_100.csv}\csva
\pgfplotstableread[col sep = comma]{csv_queries/response_time_distribution/tomcat_clean_api_validation_250.csv}\csvb
\pgfplotsset{
    /pgfplots/ybar legend/.style={
    /pgfplots/legend image code/.code={\draw[##1,/tikz/.cd,yshift=-0.25em](0cm,0cm) rectangle(1pt,0.7em);},
   }
}
\begin{tikzpicture}
  \begin{axis}[ybar, bar width=0.5, xmin = 0, ymin = 0, scaled y ticks = base 10:-5, xlabel = {Czas odpowiedzi [ms]}, ylabel = {Liczba żądań}, ymajorgrids,width=13cm, height=6cm] %TODO miary?
    \addplot[color=blue, mark=none, fill=blue] table[x index=0, y index=1]{\csva};
    \addplot[color=green, mark=none, fill=green] table[x index=0, y index=1]{\csvb};
    \legend{100,250}
  \end{axis}
\end{tikzpicture}
\caption{Tomcat 8 - rozkład czasów odpowiedzi aplikacji (95\% odpowiedzi) podczas testu walidacji istnienia klucza API}
\label{fig:tomcat_clean_api_validation_td}
\end{figure}

\pgfplotsset{grid style={dashed}}
\begin{figure}[!ht]
\pgfplotstableread[col sep = comma]{csv_queries/requests_per_sec/jetty_clean_api_validation.csv}\csvdata
\begin{tikzpicture}
  \begin{axis}[xmin = 0, xmax=900, ymin = 0, scaled y ticks = base 10:-3, xlabel = {Czas [s]}, ylabel = Liczba żądań, legend pos=south east, ymajorgrids,width=13cm, height=6cm] %TODO miary?
    \addplot[color=blue,mark=none] table[x index=0, y index=1]{\csvdata};
    \addplot[color=green,mark=none] table[x index=0, y index=2]{\csvdata};
    \legend{100,250}
  \end{axis}
\end{tikzpicture}
\caption{Jetty 9 - liczba żądań obsłużonych przez aplikację w ciągu sekundy podczas testu walidacji istnienia klucza API}
\label{fig:jetty_clean_api_validation_rps}
\end{figure}

\begin{figure}[!ht]
\pgfplotstableread[col sep = comma]{csv_queries/response_time_distribution/jetty_clean_api_validation_100.csv}\csva
\pgfplotstableread[col sep = comma]{csv_queries/response_time_distribution/jetty_clean_api_validation_250.csv}\csvb
\pgfplotsset{
    /pgfplots/ybar legend/.style={
    /pgfplots/legend image code/.code={\draw[##1,/tikz/.cd,yshift=-0.25em](0cm,0cm) rectangle(1pt,0.7em);},
   }
}
\begin{tikzpicture}
  \begin{axis}[ybar, bar width=0.5, xmin = 0, ymin = 0, scaled y ticks = base 10:-5, xlabel = {Czas odpowiedzi [ms]}, ylabel = {Liczba żądań}, ymajorgrids,width=13cm, height=6cm] %TODO miary?
    \addplot[color=blue, mark=none, fill=blue] table[x index=0, y index=1]{\csva};
    \addplot[color=green, mark=none, fill=green] table[x index=0, y index=1]{\csvb};
    \legend{100,250}
  \end{axis}
\end{tikzpicture}
\caption{Jetty 9 - rozkład czasów odpowiedzi aplikacji (95\% odpowiedzi) podczas testu walidacji istnienia klucza API}
\label{fig:jetty_clean_api_validation_td}
\end{figure}

\pgfplotsset{grid style={dashed}}
\begin{figure}[!ht]
\pgfplotstableread[col sep = comma]{csv_queries/requests_per_sec/go_clean_api_validation.csv}\csvdata
\begin{tikzpicture}
  \begin{axis}[xmin = 0, xmax=900, ymin = 0, scaled y ticks = base 10:-3, xlabel = {Czas [s]}, ylabel = Liczba żądań, legend pos=south east, ymajorgrids,width=13cm, height=6cm] %TODO miary?
    \addplot[color=blue,mark=none] table[x index=0, y index=1]{\csvdata};
    \addplot[color=green,mark=none] table[x index=0, y index=2]{\csvdata};
    \legend{100,250}
  \end{axis}
\end{tikzpicture}
\caption{Go - liczba żądań obsłużonych przez aplikację w ciągu sekundy podczas testu walidacji istnienia klucza API}
\label{fig:go_clean_api_validation_rps}
\end{figure}

\begin{figure}[!ht]
\pgfplotstableread[col sep = comma]{csv_queries/response_time_distribution/go_clean_api_validation_100.csv}\csva
\pgfplotstableread[col sep = comma]{csv_queries/response_time_distribution/go_clean_api_validation_250.csv}\csvb
\pgfplotsset{
    /pgfplots/ybar legend/.style={
    /pgfplots/legend image code/.code={\draw[##1,/tikz/.cd,yshift=-0.25em](0cm,0cm) rectangle(1pt,0.7em);},
   }
}
\begin{tikzpicture}
  \begin{axis}[ybar, bar width=0.5, xmin = 0, ymin = 0, scaled y ticks = base 10:-5, xlabel = {Czas odpowiedzi [ms]}, ylabel = {Liczba żądań}, ymajorgrids,width=13cm, height=6cm] %TODO miary?
    \addplot[color=blue, mark=none, fill=blue] table[x index=0, y index=1]{\csva};
    \addplot[color=green, mark=none, fill=green] table[x index=0, y index=1]{\csvb};
    \legend{100,250}
  \end{axis}
\end{tikzpicture}
\caption{Go - rozkład czasów odpowiedzi aplikacji (95\% odpowiedzi) podczas testu walidacji istnienia klucza API}
\label{fig:go_clean_api_validation_td}
\end{figure}


\clearpage

\subsection{Test wydajności walidacji istnienia obiektów Cache}
Wyniki testów wydajności walidacji istnienia obiektów Cache przedstawiają rysunki \ref{fig:tomcat_clean_key_validation_rps} - \ref{fig:go_clean_key_validation_td}.              

Z diagramów przedstawiających rozkład  ilości żądań obsłużonych przez poszczególne aplikacje podczas testu (rys. \ref{fig:tomcat_clean_key_validation_rps}, \ref{fig:jetty_clean_key_validation_rps}, \ref{fig:go_clean_key_validation_rps}) wynika, że największą wydajność miała aplikacja napisana w Go. Była ona w stanie obsłużyć ponad 7 tysięcy żądań na sekundę zarówno przy 100, jak i przy 250 klientach. Przy 100 klientach serwer Jetty 9 był w stanie osiągnąć przepustowość od 3.5 do 4.5 tysiąca żądań, podczas gdy serwer Tomcat 8 osiągnął najmniejsze wartości przepustowości (od 2.5 do 4 tysięcy żądań). W przypadku 250 klientów serwery Jetty 9 i Tomcat 8 osiągnęły wyniki zbliżone do siebie  (około 3.0 tysiące żądań),  jednak  serwer Tomcat 8 miał większe wahania przepustowości.

Z diagramów rozkładu czasów odpowiedzi (rys. \ref{fig:tomcat_clean_key_validation_td}, \ref{fig:jetty_clean_key_validation_td}, \ref{fig:go_clean_key_validation_td}) wynika, że podczas próby 100 klientów najlepsze wyniki osiągnęła aplikacja w Go, gdzie najdłużej trwające żądania trwały 25 milisekund, a średni czas odpowiedzi wynosił  12.80 milisekund. Rozkład czasów serwera Jetty był równomierny, a średnia czasów odpowiedzi wyniosła 21.55 milisekund. Najgorsze wyniki zanotowano w teście serwera Tomcat, gdzie średnia odpowiedzi wyniosła 26.73 milisekundy, a najdłuższe żądania trwały 120 milisekund podczas testu. W teście z 250 klientami, wyniki poszczególnych aplikacji ułożyły się w takiej samej kolejności. Najmniejszy  średni czas odpowiedzi  uzyskała  aplikacja w Go (30.43 milisekundy). Następny był serwer Jetty 9 ze średnią 76.41 milisekund, a najdłużej żądania obsługiwał serwer Tomcat 8 (średnio 80.19 milisekund). Dodatkowo, dla serwera Tomcat 8 przy 250 klientach rozkład czasów jest spłaszczony. 
\input{chapters/5_testy_wydajnosciowe_diagram_2_clean_key_validation.tex}
\clearpage

\subsection{Test wydajności operacji CRUD}
Wyniki testów wydajności operacji CRUD przedstawiają  rys. 6.13 – 6.18.

W przypadku tego testu licza żądań obsłużonych przez aplikację, zarówno przy 100 jak i 250 klientach, najlepsze wyniki miała aplikacja w Go (rys. 6.17). Obsługiwała ona od 4.5 do 5.0 tysięcy żądań w ciągu sekundy. Serwery Tomcat 8 i Jetty 9 (rys.6.13 i 6.15) wykazały się porównywalną do siebie, ale znacznie niższą przepustowością (od 2.0 do 2.5 tysiąca obsłużonych żądań). Serwer Tomcat 8 miał przy tym  większe wahania przepustowości.

\input{chapters/5_testy_wydajnosciowe_diagram_3_clean_crud.tex}
\clearpage

\subsection{Test wydajności walidacji API, obiektów Cache oraz operacji CRUD równolegle}
% \input{chapters/5_testy_wydajnosciowe_diagram_4_clean_all.tex}
\clearpage

\newpage
\section{Wyniki testów - baza danych wypełniona danymi początkowymi}
\subsection{Testy wydajnośsci walidacji API}
% \input{chapters/5_testy_wydajnosciowe_diagram_5_full_api_validation.tex}
\clearpage

\subsection{Test wydajności walidacji istnienia obiektów Cache}
% \input{chapters/5_testy_wydajnosciowe_diagram_6_full_key_validation.tex}
\clearpage

\subsection{Test wydajności operacji CRUD}
% \input{chapters/5_testy_wydajnosciowe_diagram_7_full_crud.tex}
\clearpage

\subsection{Test wydajności walidacji API, obiektów Cache oraz operacji CRUD równolegle }
% \pgfplotsset{grid style={dashed}}
\begin{figure}[!ht]
\pgfplotstableread[col sep = comma]{csv_queries/requests_per_sec/tomcat_full_all.csv}\csvdata
\begin{tikzpicture}
  \begin{axis}[xmin = 0, xmax=900, ymin = 0, scaled y ticks = base 10:-3, xlabel = {Czas [s]}, ylabel = Liczba żądań, legend pos=south east, ymajorgrids] %TODO miary?
    \addplot[color=blue,mark=none] table[x index=0, y index=1]{\csvdata};
    \addplot[color=green,mark=none] table[x index=0, y index=2]{\csvdata};
    \legend{100,250}
  \end{axis}
\end{tikzpicture}
\caption{Tomcat 8 - liczba żądań obsłużonych przez aplikację w ciągu sekundy podczas testu: walidacji istnienia klucza API, walidacji istnienia, operacji CRUD równolegle}
\label{fig:tomcat_full_all_rps}
\end{figure}

\begin{figure}[!ht]
\pgfplotstableread[col sep = comma]{csv_queries/response_time_distribution/tomcat_full_all_100.csv}\csva
\pgfplotstableread[col sep = comma]{csv_queries/response_time_distribution/tomcat_full_all_250.csv}\csvb
\pgfplotsset{
    /pgfplots/ybar legend/.style={
    /pgfplots/legend image code/.code={\draw[##1,/tikz/.cd,yshift=-0.25em](0cm,0cm) rectangle(1pt,0.7em);},
   }
}
\begin{tikzpicture}
  \begin{axis}[ybar, bar width=0.5, xmin = 0, ymin = 0, scaled y ticks = base 10:-5, xlabel = {Czas odpowiedzi [ms]}, ylabel = {Liczba żądań}, ymajorgrids] %TODO miary?
    \addplot[color=blue, mark=none, fill=blue] table[x index=0, y index=1]{\csva};
    \addplot[color=green, mark=none, fill=green] table[x index=0, y index=1]{\csvb};
    \legend{100,250}
  \end{axis}
\end{tikzpicture}
\caption{Tomcat 8 - rozkład czasów odpowiedzi aplikacji (95\% odpowiedzi) podczas testu: walidacji istnienia klucza API, walidacji istnienia, operacji CRUD równolegle}
\label{fig:tomcat_full_all_td}
\end{figure}

\pgfplotsset{grid style={dashed}}
\begin{figure}[!ht]
\pgfplotstableread[col sep = comma]{csv_queries/requests_per_sec/jetty_full_all.csv}\csvdata
\begin{tikzpicture}
  \begin{axis}[xmin = 0, xmax=900, ymin = 0, scaled y ticks = base 10:-3, xlabel = {Czas [s]}, ylabel = Liczba żądań, legend pos=south east, ymajorgrids] %TODO miary?
    \addplot[color=blue,mark=none] table[x index=0, y index=1]{\csvdata};
    \addplot[color=green,mark=none] table[x index=0, y index=2]{\csvdata};
    \legend{100,250}
  \end{axis}
\end{tikzpicture}
\caption{Jetty 9 - liczba żądań obsłużonych przez aplikację w ciągu sekundy podczas testu: walidacji istnienia klucza API, walidacji istnienia, operacji CRUD równolegle}
\label{fig:jetty_full_all_rps}
\end{figure}

\begin{figure}[!ht]
\pgfplotstableread[col sep = comma]{csv_queries/response_time_distribution/jetty_full_all_100.csv}\csva
\pgfplotstableread[col sep = comma]{csv_queries/response_time_distribution/jetty_full_all_250.csv}\csvb
\pgfplotsset{
    /pgfplots/ybar legend/.style={
    /pgfplots/legend image code/.code={\draw[##1,/tikz/.cd,yshift=-0.25em](0cm,0cm) rectangle(1pt,0.7em);},
   }
}
\begin{tikzpicture}
  \begin{axis}[ybar, bar width=0.5, xmin = 0, ymin = 0, scaled y ticks = base 10:-5, xlabel = {Czas odpowiedzi [ms]}, ylabel = {Liczba żądań}, ymajorgrids] %TODO miary?
    \addplot[color=blue, mark=none, fill=blue] table[x index=0, y index=1]{\csva};
    \addplot[color=green, mark=none, fill=green] table[x index=0, y index=1]{\csvb};
    \legend{100,250}
  \end{axis}
\end{tikzpicture}
\caption{Jetty 9 - rozkład czasów odpowiedzi aplikacji (95\% odpowiedzi) podczas testu: walidacji istnienia klucza API, walidacji istnienia, operacji CRUD równolegle}
\label{fig:jetty_full_all_td}
\end{figure}

\pgfplotsset{grid style={dashed}}
\begin{figure}[!ht]
\pgfplotstableread[col sep = comma]{csv_queries/requests_per_sec/go_full_all.csv}\csvdata
\begin{tikzpicture}
  \begin{axis}[xmin = 0, xmax=900, ymin = 0, scaled y ticks = base 10:-3, xlabel = {Czas [s]}, ylabel = Liczba żądań, legend pos=south east, ymajorgrids] %TODO miary?
    \addplot[color=blue,mark=none] table[x index=0, y index=1]{\csvdata};
    \addplot[color=green,mark=none] table[x index=0, y index=2]{\csvdata};
    \legend{100,250}
  \end{axis}
\end{tikzpicture}
\caption{Go - liczba żądań obsłużonych przez aplikację w ciągu sekundy podczas testu: walidacji istnienia klucza API, walidacji istnienia, operacji CRUD równolegle}
\label{fig:go_full_all_rps}
\end{figure}

\begin{figure}[!ht]
\pgfplotstableread[col sep = comma]{csv_queries/response_time_distribution/go_full_all_100.csv}\csva
\pgfplotstableread[col sep = comma]{csv_queries/response_time_distribution/go_full_all_250.csv}\csvb
\pgfplotsset{
    /pgfplots/ybar legend/.style={
    /pgfplots/legend image code/.code={\draw[##1,/tikz/.cd,yshift=-0.25em](0cm,0cm) rectangle(1pt,0.7em);},
   }
}
\begin{tikzpicture}
  \begin{axis}[ybar, bar width=0.5, xmin = 0, ymin = 0, scaled y ticks = base 10:-5, xlabel = {Czas odpowiedzi [ms]}, ylabel = {Liczba żądań}, ymajorgrids] %TODO miary?
    \addplot[color=blue, mark=none, fill=blue] table[x index=0, y index=1]{\csva};
    \addplot[color=green, mark=none, fill=green] table[x index=0, y index=1]{\csvb};
    \legend{100,250}
  \end{axis}
\end{tikzpicture}
\caption{Go - rozkład czasów odpowiedzi aplikacji (95\% odpowiedzi) podczas testu: walidacji istnienia klucza API, walidacji istnienia, operacji CRUD równolegle}
\label{fig:go_full_all_td}
\end{figure}


\clearpage

\section{Obciążenie maszyn wirtualnych}

\newpage
\section{Podsumowanie wyników}
