\chapter{Podsumowanie}

Przedmiotem pracy było porównanie wydajności serwisów \textsl{RESTful} w wybranych platformach programowania. Na potrzeby badań opracowano i zaimplementowano dwie identyczne aplikacje w różnych językach programowania: \textsl{Java} i \textsl{Go}. Na aplikacjach przeprowadzono testy wydajnościowe, na podstawie których oceniono ich wydajność.

Stworzone aplikacje były usługami typu \textsl{RESTful}, z którymi komunikacja odbywała się protokołem \textsl{HTTP}. Funkcje aplikacji pozwalały na zapisywanie, modyfikowanie i odczytywanie danych  zapisanych przez użytkowników. Użytkownicy autoryzowali się specjalnie dla nich wygenerowanym, unikalnym kluczem. Aplikacje do działania wykorzystywały wydajną bazę danych \textsl{MongoDB}.

W celu przeprowadzenia testów przygotowane zostało specjalne środowisko składające się z trzech maszyn wirtualnych. Najwydajniejsza z maszyn wykorzystywana była do uruchamiania przygotowanych aplikacji, druga jako baza danych, a  trzecia  służyła do uruchamiania testów wydajnościowych.

Składnie języków użytych w aplikacjach zasadniczo różnią się od siebie. Jedną z nich jest fakt, że język \textsl{Java} jest językiem obiektowym, podczas gdy \textsl{Go} pozwala na zastosowanie niewielu technik programowania obiektowego. Również obsługa błędów w obu językach jest inna. \textsl{Java} wykorzystuje do tego mechanizm rzucania i łapania wyjątków, a w \textsl{Go} obsługa błędnych sytuacji opiera się na zwracaniu z funkcji i metod obiektów typu \textsl{Error}.

Ze względu na zastosowaną technologię (\textsl{Java EE}) aplikacja w języku \textsl{Java} wymagała zastosowania kontenerów aplikacji. W celu porównania, testy przeprowadzane były z wykorzystaniem dwóch najpopularniejszych kontenerów: \textsl{Tomcat} i \textsl{Jetty}. 
 
Na podstawie  przeprowadzonych testów można jednoznacznie stwierdzić, że aplikacja w języku \textsl{Go} osiągała najlepsze rezultaty. Wyniki przepustowości tej aplikacji były wyższe od aplikacji w języku \textsl{Java} dla każdego  testowego przypadku i nie zależały od obciążenia.  Czasy odpowiedzi były również zdecydowanie krótsze, a ich rozłożenie było bardziej równomierne.    

Aplikacja w języku \textsl{Java} uruchomiona na serwerach \textsl{Tomcat} i \textsl{Jetty} charakteryzowała się dużym zróżnicowaniem przepustowości. Czasy odpowiedzi również były  dłuższe, a ich rozkład nierównomierny. Wyniki uzyskane na serwerze \textsl{Jetty} w większości przypadków były lepsze od serwera \textsl{Tomcat}. 

Aplikacja w języku \textsl{Go} wykorzystywała do działania znacznie mniej zasobów systemowych od serwerów uruchamiających aplikację  w języku \textsl{Java}. Serwer \textsl{Jetty} wymagał nieznacznie większych zasobów procesora i pamięci niż \textsl{Tomcat}. W porównaniu z aplikacją w języku \textsl{Go} oba serwery wymagały około 6 razy więcej pamięci i nieznacznie bardziej wykorzystywały procesor.

Niskie wykorzystanie zasobów systemowych przez aplikację jest ważne, jeśli architektura systemu składa się z małych usług, które są uruchamiane w rozproszonym środowisku. Dzięki temu można uruchomić wiele aplikacji, na dużo mniejszej ilości serwerów, co przekłada się na utrzymanie całego systemu. 

Podsumowując należy podkreślić, że testowanie wydajności serwisów i całych systemów jest ważnym procesem związanym z ich wytwarzaniem. Dzięki testom można określić jak szybko aplikacja odpowiada na działania użytkowników oraz jaką ich ilość jest w stanie obsłużyć równocześnie, bez zauważalnych spadków wydajności. Testy wydajnościowe pozwalają też wykryć najsłabsze punkty systemu i wyeliminować je.  Rezygnacja z testowania wydajności systemu może być bardzo kosztowna, szczególnie gdy będzie on działał produkcyjnie, a jego popularność przerośnie założenia projektantów.  
