\chapter{Wstęp}

W dobie coraz dynamiczniej rozwijającej się sieci Internet serwisy internetowe stają się częścią  naszego życia. Serwisy informacyjne, sklepy internetowe czy bankowość elektroniczna pomagają w codziennych czynnościach,  zaoszczędzając w ten sposób nasz cenny czas i wysiłek. Jednak, aby  serwisy internetowe spełniały swoje funkcje muszą  mieć odpowiednią wydajność. 

Dla użytkownika korzystającego z portali społecznościowych, system z którego korzysta wydaje się być prosty. W rzeczywistości jednak taki system jest bardzo skomplikowanym oprogramowaniem, które często działa na wielu serwerach.

W celu zapewnienia sprawnego funkcjonowania  systemu, jego niezawodności oraz umożliwienia jego ciągłego rozwoju zostało wypracowanych wiele metod inżynierii oprogramowania. Dziedzina ta wytworzyła zasady postępowania podczas tworzenia i rozwijania oprogramowania, które obejmują sposoby gromadzenia wymagań o systemie, jego implementacji, testowania jak również wdrażania. Dobrze zaprojektowany i wykonany system może przynieść ogromne zyski czasowe i finansowe. 

Wydajność jest jednym z aspektów powodzenia funkcjonowania systemu. Zaplanowanie i wykonanie odpowiednich testów wydajnościowych pozwala zbadać jak system zachowa się podczas dużego obciążenia, jak dużo użytkowników może równocześnie z niego korzystać oraz czy spełnia założone wymagania. Dzięki testom można też znaleźć i wyeliminować najsłabsze punkty systemu.

Celem niniejszej pracy jest  porównanie wydajności serwisów  \textsl{RESTful} w wybranych platformach oprogramowania. Na potrzeby badań  stworzone zostaną aplikacje typu \textsl{RESTful} zaimplementowane  w dwóch różnych językach programowania: \textsl{Java} i \textsl{Go}.  Do analizy wydajności serwisów zostaną opracowane i wykonane testy wydajnościowe, na podstawie których zostanie przeprowadzona analiza.
