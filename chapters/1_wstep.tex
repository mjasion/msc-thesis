\chapter{Wstęp}

W dobie powszechnie dostępnego internetu trudno wyobrazić sobie dziedziny życia codziennego, które jego nie wykorzystują. Miliony użytkowników codziennie wykorzystuje internet do zakupów, prowadzenia kont bankowych, przeglądania wiadomości czy też komunikacji z rodziną i znajomymi przez portale społecznościowe. 

Dla użytkownika korzystającego z portali społecznościowych, system z którego korzysta wydaje się być być prosty: wysyłanie i odbieranie wiadomości. W rzeczywistości jednak taki system jest bardzo skomplikowanym oprogramowaniem, które działa na wielu serwerach rozmieszonych często na kilku kontynentach. 

Aby system był niezawodny, pozwalał na przenoszenie oraz by był możliwy jego ciągły rozwój o nowe funkcje zostało wypracowanych wiele metod inżynierii oprogramowania. Metody te określają zasady, w jaki sposób powinno się zbierać wymagania do systemu, przez wzorce i sposoby jego implementacji i testowania, aż do sposobu jego wdrażania. System informatyczny, który zostanie przemyślany i zaprojektowany może przynieść ogromne zyski czasowe, jak i pieniężne. 

Celem niniejszej pracy jest stworzenie aplikacji typu \textsl{RESTful}, zaimplementowanych  w dwóch różnych językach programowania, na których zostaną przeprowadzone testy wydajnościowe. Do testów wybrano dwa języki: \textsl{Java} oraz \textsl{Go}. Oba języki pozwalają na uruchomienie serwera \textsl{HTTP} oraz komunikacje z bazami danych równocześnie. 