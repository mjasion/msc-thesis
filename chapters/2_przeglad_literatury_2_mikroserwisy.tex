Mikroserwisy są małymi, niezależnymi serwisami, które ze sobą współpracują\cite{newman}. Mikroserwisy są odpowiedzią na rosnącą popularność trendów w tworzeniu tworzenia oprogramowania, które są ukierunkowane na domenę(ang. \textsl{Domain-Driven Design}, skrót \textsl{DDD}), zapewniające ciągłość dostaw zmian(ang. \textsl{continous delivery}), tworzone przez małe, niezależne zespoły oraz pozwalające na ich łatwe skalowanie. 

Oprogramowanie ukierunkowane na domenę pierwszy zyskało popularność w 2003. Eric Evans w książce \textsl{Domain-Driven Design: Tackling Complexity in the Heart of Software}\cite{ddd} przedstawił sposób na tworzenie oprogramowania skupionym na domenie. Należy podkreślić że DDD nie jest metodyką tworzenia oprogramowania. Jest sposobem na zrozumienie działania rozległych systemów, co pozwala na ustalenie priorytetów podczas ich tworzenia. Książka przedstawiała to, jak ważna jest reprezentacja prawdziwego świata w tworzonym oprogramowaniu. Od 2003 roku podejście to podlega nieustannemu rozwojowi koncepcyjnemu, a jeszcze technologicznemu.

Podejście do tworzenia oprogramowania przy użyciu mikroserwisów idzie w parze z \textsl{DDD}. Mikroserwisy jak wcześniej wspomniałem są małe. Tworzone oprogramowanie rośnie wraz ze zmianami dodawanymi do systemu. Gdy system rozrośnie się, dodanie kolejnej zmiany może skutkować błędami w innej części systemu, których nie dało się przewidzieć. Często taka sytuacja zdarza się przy dużych systemach, będącymi monolitami.

Mikroserwisy przez to, że są małe pozwalają być skupione tylko na zadaniu, do którego zostały stworzone. Komunikacja między mikroserwisami odbywa się przez, sieć więc bez problemu można je uruchomić na odizolowanym środowisku lub platformie przez co mikroserwisy stają się niezależne. 

 Uruchamianie mikroserwisów na różnych systemach pozwala na stworzenie systemu bardziej odpornego na awarie. Jeśli jeden z mikroserwisów przestanie działać np. przez awarię sprzętu, spowoduje to wyłączenie pojedynczej funkcjonalności. Sam system będzie wciąż działał, a niedziałającą usługę szybko można przenieść i uruchomić w innym miejscu. Przy monolitycznym systemie, by zabezpieczyć się przed awariami musimy uruchomić ten sam system na kilku maszynach marnując przez to zasoby.

Mikroserwisy pozwalają na łatwe skalowanie. Wystarczy że uruchomimy ten sam serwis w kilku miejscach, a wydajność danej części systemu powinna wzrosnąć. W monolitycznych systemach jesteśmy zmuszeni do skalowania całych systemów więc również funkcjonalności, których w systemie rzadziej się używa również będą powielone.

Niezależność mikroserwisów jest również zaletą przy wyborze technologi. Można stworzyć te same mikroserwisy w różnych językach o ile każdy będzie działać tak samo. Przydaje się to również gdy chcemy poprawić wydajność danej usługi lub technologia, której używamy nie jest przystosowana pod konkretne zadanie. 

Kolejną sporą zaletą tworzenia systemów opartych o mikroserwisy jest wygoda ich wdrażania na środowiska produkcyjne niezależnie od pozostałych części systemu. Jeśli dodamy nową zmianę w danych serwisie, nie ma potrzeby restartowania całego systemu. Również w przypadku błędu pozwala to na szybkie znalezienie błędu, poprawienie i wdrożenie bez potrzeby zbędnego oczekiwania. 
