Mikroserwisy są małymi, niezależnymi serwisami, które ze sobą współpracują \cite{newman}. Są one odpowiedzią na rosnącą popularność trendu w tworzeniu oprogramowania ukierunkowanego na domenę (ang. \textsl{Domain-Driven Design}, skrót \textsl{DDD}) zapewniającego ciągłość dostaw zmian (ang. \textsl{continous delivery}), tworzone przez małe, niezależne zespoły oraz pozwalające na ich łatwe skalowanie. 

Tworzenie oprogramowania ukierunkowanego na domenę zyskało popularność po 2003 roku. Eric Evans w książce \textsl{Domain-Driven Design: Tackling Complexity in the Heart of Software} przedstawił sposób tworzenia takiego oprogramowania  \cite{ddd}. Należy podkreślić, że DDD nie jest metodyką tworzenia oprogramowania. Jest sposobem na zrozumienie działania rozległych systemów, co pozwala na ustalenie priorytetów podczas ich tworzenia. Autor przedstawił, jak ważna jest reprezentacja prawdziwego świata w tworzonym oprogramowaniu. 

Od 2003 roku podejście do tworzenia oprogramowania przy użyciu mikroserwisów podlega nieustannemu rozwojowi koncepcyjnemu i technologicznemu. Idzie ono w parze z \textsl{DDD}. Ponieważ mikroserwisy są małe, ich liczba rośnie wraz ze zmianami dodawanymi do systemu. W przypadku dużych,  monolitycznych systemów dodanie kolejnej zmiany może skutkować błędami w innej części systemu, których nie dało się przewidzieć. Mikroserwisy pozwalają być skupione tylko na zadaniu, do którego zostały stworzone. Komunikacja między mikroserwisami odbywa się przez sieć, więc bez problemu można je uruchomić na odizolowanym środowisku lub platformie, przez co mikroserwisy stają się niezależne. 

Uruchamianie mikroserwisów na różnych systemach pozwala na stworzenie systemu bardziej odpornego na awarie. Jeśli jeden z mikroserwisów przestanie działać np. przez awarię sprzętu, spowoduje to wyłączenie pojedynczej funkcjonalności. Sam system będzie wciąż działał, a niedziałającą usługę szybko można przenieść i uruchomić w innym miejscu. Przy monolitycznym systemie, by zabezpieczyć się przed awariami musimy uruchomić ten sam system na kilku maszynach, marnując przez to zasoby.

Mikroserwisy pozwalają na łatwe skalowanie. Wystarczy, że uruchomimy ten sam serwis w kilku miejscach, a wydajność danej części systemu powinna wzrosnąć. W monolitycznych systemach jesteśmy zmuszeni do skalowania całych systemów, również i funkcjonalności, których w systemie rzadziej się używa.

Niezależność mikroserwisów jest zaletą przy wyborze technologii. Można stworzyć te same mikroserwisy w różnych językach, o ile każdy będzie działać tak samo. Przydaje się to również, gdy chcemy poprawić wydajność danej usługi lub gdy technologia, której używamy nie jest przystosowana pod konkretne zadanie. 

Kolejną zaletą tworzenia systemów opartych o mikroserwisy jest wygoda ich wdrażania na środowiska produkcyjne niezależnie od pozostałych części systemu. Jeśli dodamy nową zmianę w danym serwisie, nie ma potrzeby restartowania całego systemu. Również w przypadku błędu pozwala to na szybkie jego znalezienie, poprawienie i wdrożenie.