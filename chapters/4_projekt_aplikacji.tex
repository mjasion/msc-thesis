\chapter{Projekt Aplikacji}
\section{Opis}
\subsection{Infrastruktura}
Do przeprowadzenia testów wydajnościowych wykorzystywane były 3 serwery wirtualne. Specyfikacje techniczne serwerów wirtualnych, wykorzystanych w testach to: 
\begin{itemize}
    \item 8 rdzeni, 16 gigabajtów pamięci RAM, 160 gigabajtów dysku SSD dla serwera aplikacyjnego
    \item 4 rdzenie, 8 gigabajtów pamięci RAM, 80 gigabajtów dysku SSD dla serwera bazy danych 
    \item 4 rdzenie, 8 gigabajtów pamięci RAM, 80 gigabajtów dysku SSD dla serwera, na którym uruchomiony był program \textsl{Apache JMeter}
\end{itemize}
Serwery komunikowały się po sieci lokalnej (ang. \textsl{LAN}) bezpośrednio między sobą.
\subsection{Testy akceptacyjne} 

By potwierdzić, że \textsl{API} aplikacji w językach \textsl{Java} i \textsl{Go} jest zachowuje się tak samo zostało przygotowane 24 testy akceptacyjnych używając biblioteki \textsl{Spock}. 

W dodatku \ref{sec:acceptance_tests_appendix} przedstawione zostały rezultaty testów aplikacji uruchomionej na serwerze \textsl{Tomcat} (rysunek \ref{fig:acceptance_test_tomcat}), \textsl{Jetty} (rysunek \ref{fig:acceptance_test_jetty}) oraz w języku \textsl{Go} (rysunek \ref{fig:acceptance_test_go}).